\chapter{List of accession types}
\label{cha:accessions}
This is a list of all the accession types (data sources) currently defined in Ondex meta-data file.
\begin{itemize}

\item{3DMET}\\ Full name: 3DMET\\ Description: A three-dimensional-structure database of natural metabolites 3DMET is a database collecting three-dimensional structures of natural metabolites - 
\url{http://www.3dmet.dna.affrc.go.jp/}

\item{AC}\\ Full name: AraCyc\\ Description: AraCyc is a metabolic pathway database for Arabidopsis thaliana that contains information about both predicted and experimentally determined pathways, reactions, compounds, genes and enzymes. - 
\url{http://www.arabidopsis.org/biocyc/}

\item{AFCS}\\ Full name: AfCS\\ Description: Alliance for Cellular Signaling - 
\url{http://www.afcs.org/}

\item{AFFYMETRIX}\\ Full name: Affymetrix\\ Description: A pioneer in genetic analysis, Affymetrix introduced highly parallel genetic assays to the marketplace by commercializing the first DNA microarray in the late 1980s. - 
\url{http://www.affymetrix.com}

\item{AHD}\\ Full name: Arabidopsis Hormone Database\\ Description: Arabidopsis Hormone Database: a comprehensive genetic and phenotypic information database for plant hormone research in Arabidopsis - Peng et al., Nucleic Acids Research 2009 37(Database issue):D975-D982; doi:10.1093/nar/gkn873 

\item{ATREGNET}\\ Full name: AtRegNet\\ Description: AtRegNet is a tool used to display the regulatory networks of Arabidopsis thaliana transcription factors. - 
\url{http://arabidopsis.med.ohio-state.edu/moreNetwork.html}

\item{BIOGRID}\\ Full name: BioGRID\\ Description: General Repository for Interaction Datasets - 
\url{http://thebiogrid.org}

\item{BKL}\\ Full name: BIOBASE Knowledge Library\\ Description: With content covering disease, drugs, gene regulation, pathways, and fully annotated genomes. - 
\url{http://www.biobase-international.com/index.php?id=databasesandtools}

\item{BRENDA}\\ Full name: Enzyme Database - BRENDA\\ Description: The Comprehensive Enzyme Information System - 
\url{http://www.brenda-enzymes.org/}

\item{BROAD}\\ Full name: Broad Institute\\ Description: Broad Institute of MIT and Harvard - 
\url{http://www.broad.mit.edu}

\item{CAMJE}\\ Full name: CAMJE\\ Description: Campylobacter jejuni proteome - 
\url{http://www.expasy.ch/sprot/hamap/CAMJE.html}

\item{CAS}\\ Full name: Chemical Abstracts Service\\ Description: CAS offers the CAS REGISTRYSM - the largest collection of substance information - as well as indexed references from more than 10,000 major scientific journals and 59 patent authorities around the world. - 
\url{http://www.cas.org/}

\item{CAZY}\\ Full name: Carbohydrate-Active enZymes\\ Description: The CAZy database describes the families of structurally-related catalytic and carbohydrate-binding modules (or functional domains) of enzymes that degrade, modify, or create glycosidic bonds. - 
\url{http://www.cazy.org/}

\item{CCSD}\\ Full name: Complex Carbohydrate Structure Database\\ Description: The Complex Carbohydrate Structure Database (CCSD) and CarbBank, an IBM PC/AT (or compatible) database management system, were created to provide an information system to meet the needs of people interested in carbohydrate science. - 
\url{http://www.boc.chem.uu.nl/sugabase/carbbank.html}

\item{CGSW}\\ Full name: Catalogue of Gene Symbols for Wheat\\ Description: A Catalogue as distributed on the MacGene CD at the 10th International Wheat Genetics Symposium. - 
\url{http://wheat.pw.usda.gov/ggpages/wgc/2003/}

\item{CHEBI}\\ Full name: Chemical Entities of Biological Interest\\ Description: Chemical Entities of Biological Interest (ChEBI) is a freely available dictionary of molecular entities focused on small chemical compounds. - 
\url{http://www.ebi.ac.uk/chebi/}

\item{CHEMBL}\\ Full name: ChEMBLdb\\ Description: ChEMBL is a database of bioactive drug-like small molecules, it contains 2-D structures, calculated properties (e.g. logP, Molecular Weight, Lipinski Parameters, etc.) and abstracted bioactivities (e.g. binding constants, pharmacology and ADMET data). - 
\url{https://www.ebi.ac.uk/chembldb/}

\item{CL}\\ Full name: Cell Ontology\\ Description: The Cell Ontology is designed as a structured controlled vocabulary for cell types. - 
\url{http://www.obofoundry.org/cgi-bin/detail.cgi?id=cell}

\item{COG}\\ Full name: Clusters of Orthologous Groups of proteins\\ Description: Clusters of Orthologous Groups of proteins (COGs) were delineated by comparing protein sequences encoded in complete genomes, representing major phylogenetic lineages. Each COG consists of individual proteins or groups of paralogs from at least 3 lineages and thus corresponds to an ancient conserved domain. - 
\url{http://www.ncbi.nlm.nih.gov/COG/}

\item{CSD}\\ Full name: Complex Carbohydrate Research Center\\ Description: The Complex Carbohydrate Research Center (CCRC) was founded at the University of Georgia (UGA) in September 1985 to answer the national need for a center devoted to increasing knowledge of the structures and functions of complex carbohydrates. - 
\url{http://www.ccrc.uga.edu/}

\item{CYORF}\\ Full name: Cyanobacteria Gene Annotation Database\\ Description: CYORF is a vehicle by which the community of cyanobacteriologists collectively annotates available genomes from cyanobacteria. - 
\url{http://cyano.genome.jp/}

\item{Corpus}\\ Full name: Corpus\\ Description: Set of publications to be analysed 

\item{DATF}\\ Full name: The Database of Arabidopsis Transcription Factors\\ Description: The Database of Arabidopsis Transcription Factors (DATF) collects all arabidopsis transcription factors (totally 1922 Loci; 2290 Gene Models) and classifies them into 64 families. - 
\url{http://datf.cbi.pku.edu.cn}

\item{DOI}\\ Full name: Digital Object Identifier\\ Description: The Digital Object Identifier (DOI) System is for identifying content objects in the digital environment. - 
\url{http://www.doi.org/}

\item{DPTF}\\ Full name: The Database of Poplar Transcription Factors\\ Description: The Database of Poplar Transcription Factors (DPTF) collects all poplar transcription factors (totally 2576 TF) and classifies them into 64 families. - 
\url{http://dptf.cbi.pku.edu.cn}

\item{DRA}\\ Full name: Drastic Insight Database\\ Description: Drastic: A Database Resource for the Analysis of Signal Transduction In Cells - 
\url{http://www.drastic.org.uk}

\item{DRUGBANK}\\ Full name: DrugBank\\ Description: The DrugBank database is a unique bioinformatics and cheminformatics resource that combines detailed drug (i.e. chemical, pharmacological and pharmaceutical) data with comprehensive drug target (i.e. sequence, structure, and pathway) information. - 
\url{http://www.drugbank.ca/}

\item{EC}\\ Full name: Enzyme Nomenclature\\ Description: SwissProt Enzyme nomenclature database ENZYME is a repository of information relative to the nomenclature of enzymes. It is primarily based on the recommendations of the Nomenclature Committee of the International Union of Biochemistry and Molecular Biology (IUBMB) and it describes each type of characterized enzyme for which an EC (Enzyme Commission) number has been provided. - 
\url{http://www.expasy.ch/enzyme/}

\item{ECOCYC}\\ Full name: EcoCyc\\ Description: Database that describes the genome and the biochemical machinery of E. coli, maintained by SRI International, Menlo Park, CA. - 
\url{http://www.ecocyc.org}

\item{EMBL}\\ Full name: EMBL-EBI International Nucleotide Sequence Data\\ Description: The EMBL Nucleotide Sequence Database (also known as EMBL-Bank) constitutes Europe's primary nucleotide sequence resource. - 
\url{http://www.ebi.ac.uk/embl/}

\item{EMBLC}\\ Full name: EMBL-EBI International Nucleotide Sequence Data Clones\\ Description: The EMBL Nucleotide Sequence Database (also known as EMBL-Bank) constitutes Europe's primary nucleotide sequence resource. - 
\url{http://www.ebi.ac.uk/embl/}

\item{ENCODE}\\ Full name: ENCyclopedia Of DNA Elements\\ Description: The National Human Genome Research Institute (NHGRI) launched a public research consortium named ENCODE, the Encyclopedia Of DNA Elements, in September 2003, to carry out a project to identify all functional elements in the human genome sequence. - 
\url{http://www.genome.gov/10005107}

\item{ENSEMBL}\\ Full name: Ensembl project\\ Description: The Ensembl project produces genome databases for vertebrates and other eukaryotic species, and makes this information freely available online. - 
\url{http://www.ensembl.org}

\item{EO}\\ Full name: Environment Ontology\\ Description: A set of standardized controlled vocabularies to describe various types of treatments given to a individual plant / a population or a cultured tissue and/or cell type sample to evaluate the response on its exposure. It also includes the study types, where the terms can be used to identify the growth study facility. Each growth facility such as field study, growth chamber, green house etc is a environment on its own it may also involve instances of biotic and abiotic environments as supplemental treatments used in these studies. - 
\url{http://www.gramene.org/plant\_ontology/ontology\_browse.html}

\item{EPD}\\ Full name: Eukaryotic Promoter Database\\ Description: The Eukaryotic Promoter Database is an annotated non-redundant collection of eukaryotic POL II promoters, for which the transcription start site has been determined experimentally. - 
\url{http://www.epd.isb-sib.ch}

\item{FLYBASE}\\ Full name: FlyBase\\ Description: A Database of Drosophila Genes and Genomes - 
\url{http://flybase.org/}

\item{GENB}\\ Full name: GenBank\\ Description: GenBank Nucleic Acid Sequence Database - 
\url{http://www.ncbi.nlm.nih.gov/Genbank/}

\item{GENEDB}\\ Full name: GeneDB\\ Description: The GeneDB project is a core part of the Sanger Institute Pathogen Sequencing Unit's (PSU) activities. - 
\url{http://www.genedb.org/}

\item{GO}\\ Full name: Gene Ontology\\ Description: The Gene Ontology project is a major bioinformatics initiative with the aim of standardizing the representation of gene and gene product attributes across species and databases. - 
\url{http://www.geneontology.org/}

\item{GOAEBI}\\ Full name: Gene Ontology Annotation\\ Description: The GOA project aims to provide high-quality Gene Ontology (GO) annotations to proteins in the UniProt Knowledgebase (UniProtKB) and International Protein Index (IPI) and is a central dataset for other major multi-species databases; such as Ensembl and NCBI. - 
\url{http://www.ebi.ac.uk/GOA/}

\item{GR}\\ Full name: Gramene\\ Description: A Resource for Comparative Grass Genomics - 
\url{http://www.gramene.org/}

\item{GeneRIF}\\ Full name: Gene Reference Into Function\\ Description: GeneRIF provides a simple mechanism to allow scientists to add to the functional annotation of genes described in Entrez Gene. - 
\url{http://www.ncbi.nlm.nih.gov/projects/GeneRIF}

\item{GrainGenes}\\ Full name: GrainGenes\\ Description: A database for Triticeae and Avena - 
\url{http://wheat.pw.usda.gov/GG2/index.shtml}

\item{HGNC}\\ Full name: HUGO Gene Nomenclature Committee\\ Description: Gene Nomenclature Committee HUGO symbols - Giving unique and meaningful names to every human gene. - 
\url{http://www.genenames.org/}

\item{HINVDB}\\ Full name: H-InvDB\\ Description: H-Invitational Database (H-InvDB) is an integrated database of human genes and transcripts. - 
\url{http://hinvdb.ddbj.nig.ac.jp/ahg-db/index.jsp}

\item{HOMSTRAD}\\ Full name: HOMologous STRucture Alignment Database\\ Description: HOMSTRAD (HOMologous STRucture Alignment Database) is a curated database of structure-based alignments for homologous protein families. - 
\url{http://tardis.nibio.go.jp/homstrad}

\item{HPRD}\\ Full name: Human Protein Reference Database\\ Description: The Human Protein Reference Database represents a centralized platform to visually depict and integrate information pertaining to domain architecture, post-translational modifications, interaction networks and disease association for each protein in the human proteome. - 
\url{http://www.hprd.org/}

\item{HSSP}\\ Full name: Homology-derived Secondary Structure of Proteins\\ Description: The HSSP database is a database of homology-derived secondary structure of proteins. - 
\url{http://swift.cmbi.kun.nl/gv/hssp/}

\item{HUGE}\\ Full name: Human Unidentified Gene-Encoded\\ Description: The HUGE protein database has been created to publicize the fruits of our Human cDNA project at the Kazusa DNA Research Institute. - 
\url{http://www.kazusa.or.jp/huge/}

\item{IAH}\\ Full name: Institute for Ageing and Health\\ Description: Institute for Ageing and Health, Newcastle University - 
\url{http://www.ncl.ac.uk/iah/}

\item{IMGT}\\ Full name: ImMunoGeneTics Database\\ Description: IMGT, the international ImMunoGeneTics project, is a collection of high-quality integrated databases specialising in Immunoglobulins, T cell receptors and the Major Histocompatibility Complex (MHC) of all vertebrate species. - 
\url{http://www.ebi.ac.uk/imgt/}

\item{INPARANOID}\\ Full name: InParanoid\\ Description: Eukaryotic Ortholog Groups. - 
\url{http://inparanoid.sbc.su.se/}

\item{INTACT}\\ Full name: IntAct\\ Description: IntAct provides a freely available, open source database system and analysis tools for protein interaction data. All interactions are derived from literature curation or direct user submissions and are freely available. - 
\url{http://www.ebi.ac.uk/intact}

\item{IPI}\\ Full name: International Protein Index\\ Description: IPI provides a top level guide to the main databases that describe the proteomes of higher eukaryotic organisms. - 
\url{http://www.ebi.ac.uk/IPI}

\item{IPRO}\\ Full name: InterPro\\ Description: A database of protein families, domains and functional sites in which identifiable features found in known proteins can be applied to unknown protein sequences. - 
\url{http://www.ebi.ac.uk/interpro}

\item{IRGSP}\\ Full name: International Rice Genome Sequencing Project\\ Description: The International Rice Genome Sequencing Project (IRGSP), a consortium of publicly funded laboratories, was established in 1997 to obtain a high quality, map-based sequence of the rice genome using the cultivar Nipponbare of Oryza sativa ssp. japonica. It is currently comprised of ten members: Japan, the United States of America, China, Taiwan, Korea, India, Thailand, France, Brazil, and the United Kingdom. The IRGSP adopts the clone-by-clone shotgun sequencing strategy so that each sequenced clone can be associated with a specific position on the genetic map and adheres to the policy of immediate release of the sequence data to the public domain. In December 2004, the IRGSP completed the sequencing of the rice genome. The high-quality and map-based sequence of the entire genome is now available in public databases. - 
\url{http://rgp.dna.affrc.go.jp/IRGSP/}

\item{JCGGDB}\\ Full name: Japan Consortium for Glycobiology and Glycotechnology DataBase\\ Description: Large-quantity synthesis of glycogenes and glycans, analysis and detection of glycan structure and glycoprotein, glycan-related differentiation markers, glycan functions, glycan-related diseases and transgenic and knockout animals, etc. - 
\url{http://jcggdb.jp/index\_en.html}

\item{JGI}\\ Full name: Joint Genome Institute\\ Description: The U.S. Department of Energy Joint Genome Institute, supported by the DOE Office of Science, unites the expertise of five national laboratories - Lawrence Berkeley, Lawrence Livermore, Los Alamos, Oak Ridge, and Pacific Northwest - along with the HudsonAlpha Institute for Biotechnology to advance genomics in support of the DOE missions related to clean energy generation and environmental characterization and cleanup. - 
\url{http://www.jgi.doe.gov/}

\item{KAZUSA}\\ Full name: Kazusa DNA Research Institute\\ Description: Japanese Genetics Institute - 
\url{http://www.kazusa.or.jp}

\item{KEGG}\\ Full name: Kyoto Encyclopedia of Genes and Genomes\\ Description: KEGG (Kyoto Encyclopedia of Genes and Genomes) is a bioinformatics resource for linking genomes to life and the environment. - 
\url{http://www.genome.jp/kegg/}

\item{KNApSAcK}\\ Full name: KNApSAcK\\ Description: KNApSAcK: A Comprehensive Species-Metabolite Relationship Database. - 
\url{http://kanaya.naist.jp/KNApSAcK/}

\item{LIGANDBOX}\\ Full name: LIGANds Data Base Open and eXtensible\\ Description: LigandBox (Ligand Data Base Open and eXtensible), a database of small chemical compounds, has been developed by Japan Biological Informatics Consortium (JBIC) as one component of the molecular simulation system, myPresto, in "Structural Proteomics Project" commissioned by the government of Japan. - 
\url{http://presto.protein.osaka-u.ac.jp/LigandBox/}

\item{LIPIDBANK}\\ Full name: LipidBank\\ Description: The official database of Japanese Conference on the Biochemistry of Lipids (JCBL). - 
\url{http://lipidbank.jp/}

\item{LIPIDMAPS}\\ Full name: Lipidomics Gateway\\ Description: Lipidomics Gateway: a free, comprehensive website for researchers interested in lipid biology. - 
\url{http://www.lipidmaps.org/}

\item{LOCUSTAG}\\ Full name: Locus tag\\ Description: Systematic submitter supplied reference number for ORF or CDS within a completed genome 

\item{LYCOCYC}\\ Full name: LycoCyc\\ Description: Developed by the Solanaceae Genomics Network, LycoCyc is a catalog of known and/or predicted biochemical pathways from tomato (Solanum lycopersicum). - 
\url{http://pathway-dev.gramene.org/gramene/lycocyc.shtml}

\item{MC}\\ Full name: MetaCyc\\ Description: MetaCyc is a database of nonredundant, experimentally elucidated metabolic pathways. MetaCyc contains more than 1400 pathways from more than 1800 different organisms, and is curated from the scientific experimental literature. - 
\url{http://metacyc.org/}

\item{MENDEL}\\ Full name: Mendel - plant gene nomenclature\\ Description: The Mendel database contains names for plant-wide families of sequenced plant genes for Commission on Plant Gene Nomenclature (CPGN). - 
\url{http://www.ncbi.nlm.nih.gov/pmc/articles/PMC29857/}

\item{MEROPS}\\ Full name: MEROPS\\ Description: The MEROPS database is an information resource for peptidases (also termed proteases, proteinases and proteolytic enzymes) and the proteins that inhibit them. - 
\url{http://merops.sanger.ac.uk/}

\item{MGI}\\ Full name: Mouse Genome Informatics\\ Description: MGI: the international database resource for the laboratory mouse, providing integrated genetic, genomic, and biological data for researching human health. - 
\url{http://www.informatics.jax.org/}

\item{MINT}\\ Full name: Molecular INTeraction database\\ Description: MINT focuses on experimentally verified protein-protein interactions mined from the scientific literature by expert curators. - 
\url{http://mint.bio.uniroma2.it/mint/}

\item{MIPS}\\ Full name: Munich Information Center for Protein Sequences\\ Description: The Munich Information Center for Protein Sequences (MIPS) is a research center hosted at the Institute for Bioinformatics (IBI) at Neuherberg, Germany with a focus on genome oriented bioinformatics, in particular on the systematic analysis of genome information including the development and application of bioinformatics methods in genome annotation, gene expression analysis and proteomics. - 
\url{http://mips.gsf.de/}

\item{MIRBASE}\\ Full name: miRBase - the microRNA database\\ Description: The miRBase database is a searchable database of published miRNA sequences and annotation. Each entry in the miRBase Sequence database represents a predicted hairpin portion of a miRNA transcript (termed mir in the database), with information on the location and sequence of the mature miRNA sequence (termed miR). - 
\url{http://microrna.sanger.ac.uk}

\item{MODBASE}\\ Full name: Database of Comparative Protein Structure Models\\ Description: MODBASE is a queryable database of annotated protein structure models. - 
\url{http://modbase.compbio.ucsf.edu}

\item{MSDchem}\\ Full name: Macromolecular Structure Database Group\\ Description: MSDchem: Consistent and enriched library of ligands, small molecules and monomers that are referred as residues and hetgroups in any PDB entry. - 
\url{http://www.ebi.ac.uk/msd-srv/msdchem}

\item{MeSH}\\ Full name: Medline Subject Headings\\ Description: MeSH is the National Library of Medicine's controlled vocabulary thesaurus. It consists of sets of terms naming descriptors in a hierarchical structure that permits searching at various levels of specificity. - 
\url{http://www.nlm.nih.gov/mesh/}

\item{NCBI}\\ Full name: National Center for Biotechnology Information\\ Description: National Center for Biotechnology Information, National Library of Medicine, National Institutes of Health. - 
\url{http://www.ncbi.nlm.nih.gov/}

\item{NCI}\\ Full name: NCI Chemical Carcinogen Repository\\ Description: The National Cancer Institute (NCI) contracts with MRI to operate a repository that supplies standard reference-grade compounds for cancer research. - 
\url{http://www.mriresearch.org/WorkingWithMRI/NCIRepository.asp}

\item{NC\_GE}\\ Full name: NCBI GE GeneID database\\ Description: These GeneID values are specific for the ENTREZGENE database at the National Center for Biotechnology Information (THIS IS ABSOLUTELY NOT THE SAME AS: NC\_GI) - 
\url{http://www.ncbi.nlm.nih.gov/gene}

\item{NC\_GI}\\ Full name: NCBI GI GeneID database\\ Description: National Center for Biotechnology Information Gene ID - 
\url{http://www.ncbi.nlm.nih.gov/entrez/query/static/help/genefaq.html}

\item{NC\_NM}\\ Full name: NCBI DNA/RNA sequence database\\ Description: NCBI DNA/RNA sequence from REFSEQ (National Center for Biotechnology Information) - 
\url{http://www.ncbi.nlm.nih.gov/RefSeq/}

\item{NC\_NP}\\ Full name: NCBI protein sequence database\\ Description: NCBI protein sequence from REFSEQ (National Center for Biotechnology Information) - 
\url{http://www.ncbi.nlm.nih.gov/RefSeq/}

\item{NEWT}\\ Full name: UniProt Taxonomy Database\\ Description: NEWT is the taxonomy database maintained by the UniProt group. It integrates taxonomy data compiled in the NCBI database and data specific to the UniProt Knowledgebase. - 
\url{http://www.uniprot.org/help/taxonomy}

\item{NIKKAJI}\\ Full name: Japan Chemical Substance Dictionary\\ Description: An organic compound dictionary database prepared by the Japan Science and Technology Agency (JST). - 
\url{http://nikkajiweb.jst.go.jp/nikkaji\_web/pages/top\_e.html}

\item{NITE}\\ Full name: National Institute for Technology and Evaluation\\ Description: National Institute for Technology and Evaluation - 
\url{http://www.nite.go.jp/index-e.html}

\item{NLM}\\ Full name: National Library Of Medicine\\ Description: The Library collects materials and provides information and research services in all areas of biomedicine and health care. - 
\url{http://www.nlm.nih.gov/}

\item{OMIM}\\ Full name: Online Mendelian Inheritance in Man\\ Description: OMIM, Online Mendelian Inheritance in Man, a database of human genes and genetic disorders developed by staff at Johns Hopkins and hosted on the Web. - 
\url{http://www.ncbi.nlm.nih.gov/omim}

\item{Oryzabase}\\ Full name: Oryzabase\\ Description: The Oryzabase is a comprehensive rice science database established in 2000 by rice researcher's committee in Japan. The database is originally aimed to gather as much knowledge as possible ranging from classical rice genetics to recent genomics and from fundamental information to hot topics. - 
\url{http://www.shigen.nig.ac.jp/rice/oryzabase/top/top.jsp}

\item{PATHODB}\\ Full name: BIOBASE PathoDB\\ Description: PathoDB is a database on pathologically relevant mutated forms of transcription factors and their binding sites. - 
\url{http://www.biobase-international.com/index.php?id=pathodb}

\item{PDB}\\ Full name: Protein Data Bank\\ Description: A Resource for Studying Biological Macromolecules - The PDB archive contains information about experimentally-determined structures of proteins, nucleic acids, and complex assemblies. - 
\url{http://www.pdb.org}

\item{PFAM}\\ Full name: Pfam\\ Description: Pfam is a large collection of multiple sequence alignments and hidden Markov models covering many common protein domains and families. - 
\url{http://pfam.sanger.ac.uk/}

\item{PHI}\\ Full name: PHI-base\\ Description: Pathogen - Host Interaction database - 
\url{http://www.phibase.org}

\item{PHYTOZOME}\\ Full name: Phytozome\\ Description: Phytozome is a joint project of the Department of Energy's Joint Genome Institute and the Center for Integrative Genomics to facilitate comparative genomic studies amongst green plants. - 
\url{http://www.phytozome.net}

\item{PHYTOZOME-POPLAR}\\ Full name: Phytozome Poplar DB\\ Description: Poplar specific data on Phytozome which is a joint project of the Department of Energy's Joint Genome Institute and the Center for Integrative Genomics to facilitate comparative genomic studies amongst green plants. - 
\url{http://www.phytozome.net}

\item{PIR}\\ Full name: Protein Information Resource\\ Description: The Protein Information Resource (PIR) is an integrated public bioinformatics resource to support genomic, proteomic and systems biology research and scientific studies. - 
\url{http://pir.georgetown.edu}

\item{PLANTCYC}\\ Full name: Plant Metabolic Pathway Database\\ Description: PlantCyc provides access to manually curated or reviewed information about shared and unique metabolic pathways present in over 250 plant species. - 
\url{http://www.plantcyc.org/}

\item{PO}\\ Full name: Plant Ontology\\ Description: Ontologie that describe plant structures and growth and developmental stages, providing a semantic framework for meaningful cross-species queries across databases. - 
\url{http://www.plantontology.org/}

\item{PRINTS}\\ Full name: Protein fingerprints\\ Description: A compendium of protein fingerprints. A fingerprint is a group of conserved motifs used to characterise a protein family; its diagnostic power is refined by iterative scanning of a SWISS-PROT/TrEMBL composite. - 
\url{http://www.bioinf.manchester.ac.uk/dbbrowser/PRINTS}

\item{PRODOM}\\ Full name: ProDom\\ Description: ProDom is a comprehensive set of protein domain families automatically generated from the SWISS-PROT and TrEMBL sequence databases. - 
\url{http://prodom.prabi.fr/prodom/current/html/home.php}

\item{PROID}\\ Full name: DDBJ/EMBL-bank/GenBank databases\\ Description: The protein identifier shared by DDBJ/EMBL-bank/GenBank nucleotide (e.g. CAA71991) 

\item{PROSITE}\\ Full name: ExPASy Proteomics Server\\ Description: PROSITE consists of documentation entries describing protein domains, families and functional sites as well as associated patterns and profiles to identify them. - 
\url{http://www.expasy.ch/prosite}

\item{PSI-MIO}\\ Full name: PSI-MI Ontology\\ Description: Proteomics Standards Initiative - Molecular Interactions Ontology. - 
\url{http://www.psidev.info/}

\item{PUBCHEM}\\ Full name: PubChem\\ Description: Free database of chemical structures of small organic molecules and information on their biological activities. - 
\url{http://pubchem.ncbi.nlm.nih.gov/}

\item{PlnTFDB}\\ Full name: Plant TFDB\\ Description: Plant Transcription factor databasePlnTFDB (2.0) is a public database arising from efforts to identify and catalogue all Plant genes involved in transcriptional control. - 
\url{http://plntfdb.bio.uni-potsdam.de/v2.0/}

\item{Poplar-JGI}\\ Full name: Poplar DB at JGI\\ Description: Poplar database at JGI which is the U.S. Department of Energy Joint Genome Institute, - 
\url{http://www.jgi.doe.gov/}

\item{PubMed}\\ Full name: PubMed\\ Description: PubMed comprises more than 19 million citations for biomedical articles from MEDLINE and life science journals. Citations may include links to full-text articles from PubMed Central or publisher web sites. - 
\url{http://www.ncbi.nlm.nih.gov/pubmed/}

\item{RATMAP}\\ Full name: The Rat Genome Database. RATMAP\\ Description: The Rat Genome Database RatMap is focused on presenting rat genes, DNA-markers, QTLs etc that is localized to chromosome. - 
\url{http://ratmap.gen.gu.se/}

\item{REAC}\\ Full name: Reactome\\ Description: A curated knowledgebase of biological pathways. - 
\url{http://www.reactome.org}

\item{RESID}\\ Full name: RESID - Database of Protein Modifications\\ Description: The RESID Database of Protein Modifications is a comprehensive collection of annotations and structures for protein modifications including amino-terminal, carboxyl-terminal and peptide chain cross-link post-translational modifications. - 
\url{http://www.ebi.ac.uk/RESID/}

\item{RGD}\\ Full name: Rat Genome Database\\ Description: The Rat Genome Database is a collaborative effort between leading research institutions involved in rat genetic and genomic research. - 
\url{http://rgd.mcw.edu/}

\item{RHEA}\\ Full name: RHEA - Reaction Database\\ Description: Rhea is a reaction database, where all reaction participants (reactants and products) are linked to the ChEBI database (Chemical Entities of Biological Interest). - 
\url{http://www.ebi.ac.uk/rhea/}

\item{RSNP}\\ Full name: Repository of Single Nucleotide Polymorphism\\ Description: A system of databases documenting the influence of mutations in regulatory gene regions onto DNA interaction with nuclear proteins. - 
\url{http://wwwmgs.bionet.nsc.ru/mgs/systems/rsnp}

\item{SA}\\ Full name: Sanger Institute\\ Description: The Sanger Institute is a genome research institute primarily funded by the Wellcome Trust. - 
\url{http://www.sanger.ac.uk/}

\item{SCOP}\\ Full name: Structural Classification of Proteins\\ Description: The SCOP database, created by manual inspection and abetted by a battery of automated methods, aims to provide a detailed and comprehensive description of the structural and evolutionary relationships between all proteins whose structure is known. - 
\url{http://scop.mrc-lmb.cam.ac.uk/scop/}

\item{SGD}\\ Full name: Saccharomyces Genome Database\\ Description: SGD is a scientific database of the molecular biology and genetics of the yeast Saccharomyces cerevisiae, which is commonly known as baker's or budding yeast. - 
\url{http://www.yeastgenome.org/}

\item{SGN}\\ Full name: Sol Genomics Network\\ Description: See webpage 
\url{http://solgenomics.net}

\item{SMART}\\ Full name: Simple Modular Architecture Research Tool\\ Description: SMART (a Simple Modular Architecture Research Tool) allows the identification and annotation of genetically mobile domains and the analysis of domain architectures. More than 500 domain families found in signalling, extracellular and chromatin-associated proteins are detectable. - 
\url{http://smart.embl-heidelberg.de/}

\item{SMARTDB}\\ Full name: S/MAR transaction DataBase\\ Description: S/MARt DB collects information about scaffold/matrix attached regions and the nuclear matrix proteins that are supposed be involved in the interaction of these elements with the nuclear matrix. - 
\url{http://smartdb.bioinf.med.uni-goettingen.de}

\item{SMOD}\\ Full name: SWISS-MODEL\\ Description: SWISS-MODEL is a fully automated protein structure homology-modeling server, accessible via the ExPASy web server, or from the program DeepView (Swiss Pdb-Viewer). - 
\url{http://swissmodel.expasy.org/}

\item{SO}\\ Full name: Sequence Ontology\\ Description: The Sequence Ontology is a set of terms and relationships used to describe the features and attributes of biological sequence. - 
\url{http://www.sequenceontology.org/}

\item{SOYCYC}\\ Full name: SoyCyc\\ Description: This is the first version of the SoyCyc Soybean Metabolic Pathway Database. - 
\url{http://www.soybase.org:8082/}

\item{TAIR}\\ Full name: The Arabidopsis Information Resource\\ Description: The Arabidopsis Information Resource (TAIR) maintains a database of genetic and molecular biology data for the model higher plant Arabidopsis thaliana. - 
\url{http://www.arabidopsis.org/}

\item{TC}\\ Full name: Transport Classification Database\\ Description: The TC-DB website details a comprehensive classification system for membrane transport proteins known as the Transport Classification (TC) system. - 
\url{http://megaman.ucsd.edu/tcdb}

\item{TF}\\ Full name: TRANSFAC\\ Description: BIOBASE TRANSFAC Professional: The most comprehensive collection of eukaryotic gene regulation data - 
\url{http://www.biobase-international.com/pages/index.php?id=transfac}

\item{TIGR}\\ Full name: The Institute for Genomic Research\\ Description: The Institute for Genomic Research (TIGR) was a non-profit genomics research institute founded in 1992 by Craig Venter in Rockville, Maryland, United States. It is now a part of the J. Craig Venter Institute. - 
\url{http://www.tigr.org}

\item{TO}\\ Full name: Trait Ontology\\ Description: It is a controlled vocabulary to describe each trait as a distinguishable feature, characteristic, quality or phenotypic feature of a developing or mature individual. Examples are glutinous endosperm, disease resistance, plant height, photosensitivity, male sterility, etc. - 
\url{http://www.gramene.org/plant\_ontology/ontology\_browse.html#to}

\item{TP}\\ Full name: TRANSPATH\\ Description: BIOBASE Transpath Database: TRANSPATH is a database system about gene regulatory networks that combines encyclopedic information on signal transduction with tools for visualization and analysis. - 
\url{http://www.biobase.de/}

\item{TRANSCOMPEL}\\ Full name: TRANSCompel\\ Description: Composite regulatory elements are found in many promoters and enhancers of eukaryotic genes. They consist of two binding sites of two different transcription factors, which through this combination form a module with new regulatory properties. Composite elements frequently serve as integration sites of two (or more) signaling pathways. - 
\url{http://www.biobase-international.com/pages/index.php?id=transcompel}

\item{TRANSPRO}\\ Full name: TRANSPro\\ Description: Promoter sequences are defined according to the annotated transcription start sites (TSS), which are retrieved and clustered by a proprietary algorithm to distinguish alternative promoters of individual genes. TRANSPro is a necessary resource for systematic promoter analysis of co-regulated genes. Known transcription factor binding sites are indicated. - 
\url{http://www.biobase-international.com/index.php?id=transpro}

\item{TRRD}\\ Full name: Transcription Regulatory Regions Database\\ Description: TRRD is a unique information resource, accumulating information on structural and functional organization of transcription regulatory regions of eukaryotic genes. Only experimentally confirmed information is included into TRRD. - 
\url{http://wwwmgs.bionet.nsc.ru/mgs/gnw/trrd}

\item{UC}\\ Full name: unclassified data source\\ Description: The data source cannot be classified. This is for example the case for concepts merged together from different CVs. 

\item{UM-E}\\ Full name: UM-E\\ Description: UM-BBD\_enzymeID 

\item{UM-P}\\ Full name: UM-P\\ Description: UM-BBD\_pathwayID 

\item{UM-R}\\ Full name: UM-BBD\_reactionID\\ Description: UM-BBD\_reactionID 

\item{UMBBD}\\ Full name: University of Minnesota Biocatalysis/Biodegradation Database\\ Description: This database contains information on microbial biocatalytic reactions and biodegradation pathways for primarily xenobiotic, chemical compounds. - 
\url{http://umbbd.msi.umn.edu}

\item{UMan}\\ Full name: University of Manchester\\ Description: Britain's largest single-site university with a proud history of achievement and an ambitious agenda for the future. - 
\url{http://www.manchester.ac.uk/}

\item{UNIGENE}\\ Full name: UniGene - An Organized View of the Transcriptome\\ Description: Each UniGene entry is a set of transcript sequences that appear to come from the same transcription locus (gene or expressed pseudogene), together with information on protein similarities, gene expression, cDNA clone reagents, and genomic location. - 
\url{http://www.ncbi.nlm.nih.gov/entrez/query.fcgi?db=unigene}

\item{UNIPARC}\\ Full name: UniParc\\ Description: UniProt Archive (UniParc) is part of UniProt project. It is a non-redundant archive of protein sequences extracted from public databases UniProtKB/Swiss-Prot, UniProtKB/TrEMBL, PIR-PSD, EMBL, EMBL WGS, Ensembl, IPI, PDB, PIR-PSD, RefSeq, FlyBase, WormBase, H-Invitational Database, TROME database, European Patent Office proteins, United States Patent and Trademark Office proteins (USPTO) and Japan Patent Office proteins. - 
\url{http://www.ebi.ac.uk/uniparc/}

\item{UNIPROTKB}\\ Full name: UniProtKB\\ Description: UniProtKB is protein sequence database consisting oftwo sections 1. Swiss-Prot, which is manually annotated and reviewed; 2. TrEMBL, which is automatically annotated and not reviewed - 
\url{http://www.uniprot.org/}

\item{UNIPROTKB-SwissProt}\\ Full name: UniProtKB/Swiss-Prot\\ Description: UniProtKB/Swiss-Prot is a reviewed and manually-annotated protein sequence database. - 
\url{http://www.uniprot.org/}

\item{UNIPROTKB-TrEMBL}\\ Full name: UniProtKB/TrEMBL\\ Description: UniProtKB/TrEMBL is a computer-annotated and not reviewed protein sequence database. - 
\url{http://www.uniprot.org/}

\item{UWASH}\\ Full name: University of Washington Multimegabase Sequencing Center\\ Description: Washington University School of Medicine in St. Louis - 
\url{http://www.genome.washington.edu/uwgc/}

\item{VEGA}\\ Full name: The Vertebrate Genome Annotation database\\ Description: The Vertebrate Genome Annotation (VEGA) database is a central repository for high quality manual annotation of vertebrate finished genome sequence. - 
\url{http://vega.sanger.ac.uk}

\item{VO}\\ Full name: Virtual Ontology\\ Description: Virtual Ontology, a temporary ontology with no permanent representation outside ONDEX. 

\item{WB}\\ Full name: WormBase\\ Description: WormBase is an international consortium of biologists and computer scientists dedicated to providing the research community with accurate, current, accessible information concerning the genetics, genomics and biology of C. elegans and some related nematodes. - 
\url{http://www.wormbase.org}

\item{WIKI}\\ Full name: Wikipedia\\ Description: Wikipedia, the free encyclopedia that anyone can edit. - 
\url{http://wikipedia.org}

\item{WN}\\ Full name: WordNet\\ Description: WordNet is a large lexical database of English, developed under the direction of George A. Miller. Nouns, verbs, adjectives and adverbs are grouped into sets of cognitive synonyms (synsets), each expressing a distinct concept. - 
\url{http://wordnet.princeton.edu/}

\item{ZFIN}\\ Full name: The Zebrafish Information Network\\ Description: ZIRC is the Zebrafish International Resource Center, an independent NIH-funded facility providing a wide range of zebrafish lines, probes and health services. - 
\url{http://zfin.org}

\item{dbEST}\\ Full name: dbEST\\ Description: dbEST (Nature Genetics 4:332-3;1993) is a division of GenBank that contains sequence data and other information on "single-pass" cDNA sequences, or Expressed Sequence Tags, from a number of organisms. - 
\url{http://www.ncbi.nlm.nih.gov/dbEST/}

\item{maizeGDB}\\ Full name: Maize Genetics and Genomics Database\\ Description: MaizeGDB is the community database for biological information about the crop plant Zea mays ssp. mays. Genetic, genomic, sequence, gene product, functional characterization, literature reference, and person/organization contact information are among the datatypes accessible through this site. - 
\url{http://www.maizegdb.org/}

\item{unknown}\\ Full name: unknown data source\\ Description: When the data source is unknown, e.g. data parsed from excel spreadsheets not part of a particular database or consortium. 

\end{itemize}
