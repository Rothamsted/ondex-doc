%%%%%%%%%%%%%%%%%%%%%%%%%%%%%%%%%%%%%%%%%%%%%%%%%%%%%%%%%%%%%%%%%%%%%%%%%%%%%%%%%%%%%%%%%%%%%%%%%%%%%%%%%%%%%%%%%%%%%%%%%%%%%%%%%%%%%%%%%%%%%%%%%%%%%%%%%%%%%%%%%%%%
\chapter{Mapping data in Ondex}
\label{cha:map}
%%%%%%%%%%%%%%%%%%%%%%%%%%%%%%%%%%%%%%%%%%%%%%%%%%%%%%%%%%%%%%%%%%%%%%%%%%%%%%%%%%

Mapping methods allow users to create ``equ'' relations (or what you specify as their name) between concepts identified as equivalent
based on a specified property, whether accession, name, sequence, attribute value, {\it{etc}}.

%%%%%%%%%%%%%%%%%%%%%%%%%%%%%%%%%%%%%%%%%%%%%%%%%%%%%%%%%%%%%%%%%%%%%%%%%%%%%%%%%%

\section{Concept accession-based mapping (Memory-efficient)} 
\label{sec:accmapping}
Accession based mapping maps all concepts of the same concept class ({\em{CC}})
and of different data source or controlled vocabulary ({\em{CV}}) unless stated otherwise.	
See examples in Section \ref{sec:accmapping_examples}.
	
\begin{itemize}
  
  \item{EquivalentConceptClass (optional)}\\
  If you want your accession based mapping to map concepts across CCs rather than within CCs, 
  this is the option you need to fill in.
  This option should contain a pair of CCs seperated by a comma ({\it{e.g.}}, ``Target,Protein'').
  
  \item{EquivalentCV (optional)}\\
  EquivalentCV means equivalent accession type.
  This option is on the data source of the accession of the concept (not the data source of the concept).
  The usage of this setting is to explicitly tell that two accessions with different accession types actually contain the same information.
  This option should contain a pair of data sources seperated by a comma ({\it{e.g.}}, ``TIGR,TAIR'').
  
  \item{AttributeRestriction (optional)}\\
  This will limit the mapping method to only map concepts with an attribute value matching
  the attribute name specified by this parameter.
  
  \item{IgnoreAmbiguity (optional)}\\
  IgnoreAmbiguity means use ambiguous accessions.
  When true, this allows ambiguous concept accessions to be mapped.
  
  \item{RelationType (optional)}\\
  This mapping method will create relations between concepts that map.
  This option specifies the relation type of the relation to be created.
  By default, the relation type ``equals'' (equ) is used.
  
  \item{WithinCVMapping (optional)}\\
  WithinCVMapping means within data source.
  Instead of mapping across all data sources, the mapping will be performed within data source(s) specified here.
  
  \item{ConceptClassRestriction (optional)}\\
  This is the option to use if you wish to restrict the mappings to concepts within one or more specified concept classe(s).
  For example, if a network contains genes and proteins but you only want to map proteins between themselves 
  (rather than proteins between themselves and genes between themselves), you would use this option specifying ``Protein''.
  
  \item{CVRestriction (optional)}\\
  CVRestriction means accession type restriction.
  This restriction is on the data source of the accession of the concept (not the data source of the concept).

\end{itemize}
    

%%%%%%%%%%%%%%%%%%%%%%%%%%%%%%%%%%%%%%%%%%%%%%%%%%%%%%%%%%%%%%%%%%%%%%%%%%%%%%%%%%
\section{BLAST based mapping}
This mapping uses the Basic Local Alignment Search Tool (BLAST) to search for similar sequences.
\begin{itemize}
  
  \item{EquivalentConceptClass (optional)}\\
  If you want your BLAST based mapping to map concepts across concept classes rather than within concept classes, 
  this is the option you need to fill in.
  This option should contain a pair of concept classes seperated by a comma ({\it{e.g.}}, ``Target,Protein'').
  
  \item{PathToBlast (required)}\\
  Specify the path to your BLAST+ executable.
  
  \item{Evalue (optional)}\\
  E-value cutoff BLAST argument (\url{http://www.swbic.org/origin/proc_man/Blast/BLAST_tutorial.html}).
  
  \item{SeqAttribute (optional)}\\
  Specifies the attribute containing the sequence data.
  
  \item{SeqType (optional)}\\
  Specifies what sequence type is contained in the attribute ({\it{e.g.}}, NA, AA).
  
  \item{WithinSameDataSource (optional)}\\
  This option allows mapping within the same data source.
   
\end{itemize}


%%%%%%%%%%%%%%%%%%%%%%%%%%%%%%%%%%%%%%%%%%%%%%%%%%%%%%%%%%%%%%%%%%%%%%%%%%%%%%%%%%    
\section{Cross species sequence mapping}
The cross-species mapping method calculates the best orthologs for each species present in the graph for each of the query concepts.
The best ortholog in this instance is defined as the largest coverage of the shortest sequence.
An example usage is when annotating a micro-array to find the closest orthologs for each sequence in each database.

\begin{itemize}
  
  \item{PerSpeciesBlast (optional)}\\
  Set to false (default is true) to create for each all qualifying species in the graph only blast DB along with an one decypher BLAST job. 
  (beware e-values may be higher given the increase in abundence of sequence hits for a given gene family) 
  This is very slow, however it will reduce the DB size for each BLAST job and improve the E-value of the hits.
  
  \item{Overlap (optional)}\\
  The minimum overlap to tolerate. The default value is 0.25.
  
  \item{QueryConceptClass (optional)}\\
  Filter by a target Query CC (default is all).
  
  \item{SeqDBSize (optional)}\\
  The number of sequences requires in a species to qualify for blasting against (Useful if PerSpeciesBlast is off)
  
  \item{QuerySequenceType (required)}\\
  Sequence type must inherit from either Amino Acid (AA) or Nucleic Acid (NA).
  
  \item{EntryPointIsCV (optional)}\\
  Indicates whether CV (database) should be considered a unique entry point for blasting.
  This forces the algorithm to identify the best orthologs for each species and database (CV).
  {\it{E.g.}}, it is possible to find the best orthologs for each species and each database (KEGG, AraCyc, TransFac).
  
  \item{AlignmentsCutoff (optional)}\\
  The minimum length of sequence to allow as a valid alignment.
  
  \item{TargetSequenceType (required)}\\
  Sequence type must inherit from either Amino Acid (AA) or Nucleic Acid (NA).
  
  \item{AlignmentsPerQuery (optional)}\\
  Maximum alignments decypher will return per query
  
  \item{ComparisonType (optional)}\\	  
  Comparison type that defines the better alignment (bitscore, covera).
  
  \item{EntryPointIsTAXID (optional)}\\
  Indicates whether species should be considered a unique entry point for blasting.
  
  \item{AlignmentThreshold (optional)}\\
  Indicates the number (integer parameter) of alignments that should be taken per entry point, 
  based on the top ranking coverage/alignment length of the longer/shorter(default) sequence in the match.
  The default value is 1.
  
  \item{ProgramDir (required)}\\
  Specify the directory where the sequence alignment program is located.
  
  \item{Bitscore (required)}\\
  The minimum bitscore to tolerate
  
  \item{EValue (required)}\\
  Cutoff for the BLAST E-Value (\url{http://www.swbic.org/origin/proc_man/Blast/BLAST_tutorial.html}).
  This option reduces the search space for identifying orthologs.
  It does not mean the algorithm searches for the best E-value rather than the best alignment.
  
  \item{TargetConceptClass (optional)}\\
  Filter by a target concept class.
  
  \item{QueryTAXID (required)}\\
  TaxID for which to perform the cross species mapping.

\end{itemize}


%%%%%%%%%%%%%%%%%%%%%%%%%%%%%%%%%%%%%%%%%%%%%%%%%%%%%%%%%%%%%%%%%%%%%%%%%%%%%%%%%%    
\section{Attribute equality mapping}
Creates relation where all specified concept attributes are equal.

\begin{itemize}
  
  \item{AttributeNames (required)}\\
  AttributeNames that must be present and equal for concept to map.
  
  \item{RelationType (required)}\\
  RelationType to create when conditions are met.
  
  \item{EquivalentConceptClass (optional)}\\
  This option should contain a pair of ConceptClasses seperated by a comma (,); for example "Thing,EC" 
  The usage of this setting is to allow the mapping method to cross the ConceptClass boundary in some special cases. 
  And thus be able to map for example similar GO and EC concepts to each other.
  
  \item{IgnoreCase (optional)}\\
  Ignore case if comparing Strings
  
  \item{WithinDataSourceMapping (optional)}\\
  Map within data sources.
  
  \item{AttributeRestriction (optional)}\\
  This will limit the mapping method to only map concepts when the Attribute Value with the attribute name specified by this parameter is the same.
  
  \item{ReplacePattern (optional)}\\
  An optional preprocessing pattern if Attribute is on Strings
  
  \item{ConceptClass (optional)}\\
  ConceptClass to restrict mapping to

\end{itemize}


%%%%%%%%%%%%%%%%%%%%%%%%%%%%%%%%%%%%%%%%%%%%%%%%%%%%%%%%%%%%%%%%%%%%%%%%%%%%%%%%%%    
\section{Inparanoid}
Implements the INPARANOID algorithm\footnote{Automatic clustering of orthologs and in-paralogs from pairwise species comparisons.
Rmm M., Storm C.E., Sonnhammer E.L. Journal of molecular biology, 314(5):1041-52, PubMed ID: 11743721} as a mapping method for the Ondex system.

\begin{itemize}
  
  \item{PathToBlast (required)}\\
  Path to BLAST+ executable.
  
  \item{Evalue (optional)}\\
  Evalue cutoff BLAST argument.
  
  \item{SeqAttribute (optional)}\\
  Specifies the attribute containing the sequence data.
  The default is AA.
  
  \item{SeqType (optional)}\\
  Specifies what sequence type is contained in the attribute ({\it{e.g.}}, ``NA, AA'').
  
  \item{Cutoff (optional)}\\
  Bit-score cutoff (default 30).
  
  \item{Overlap (optional)}\\
  Sequence overlap of match length compared to longest sequences (default 0.5).
 
\end{itemize}
    
%%%%%%%%%%%%%%%%%%%%%%%%%%%%%%%%%%%%%%%%%%%%%%%%%%%%%%%%%%%%%%%%%%%%%%%%%%%%%%%%%%    
\section{Interolog Mapping}
Creates predicted protein-protein interaction relations by inferring interactions 
from interactions in other species based on ortholog and paralog relations.
\begin{itemize}

  \item{AttributeRestriction (optional)}\\
  Limits mapping to concepts with equivalent attribute values (for the attribute name specified),
  {\it{e.g.}} taxonomy id to ensure concepts are from the same species.

\end{itemize}

%%%%%%%%%%%%%%%%%%%%%%%%%%%%%%%%%%%%%%%%%%%%%%%%%%%%%%%%%%%%%%%%%%%%%%%%%%%%%%%%%%    
\section{ConceptName based mapping}
Implements a ConceptName based mapping which is always case insensitive whatever options are ticked.

\begin{itemize}
  
  \item{AttributeRestriction (optional)}\\
  This will limit the mapping method to only map concepts when the attribute value with the attribute name
  specified by this parameter is the same.
  
  \item{EquivalentConceptClass (optional)}\\
  This option should contain a pair of concept classes seperated by a comma ({\it{e.g.}}, ``Thing,EC'').
  The usage of this setting is to allow the mapping method to cross the concept class boundary in some special cases
  and thus be able to map for example similar GO and EC concepts to each other.
  
  \item{ExactSynonyms (optional)}\\
  Matches preferred concept names or synonyms (shown in green in the item information window) only.
  
  \item{NameThreshold (required)}\\
  By default two names/synonyms of two concepts must match for them to be mapped.
  This is set so as to avoid mapping ambiguous names.
  However the threshold can be set to any number, 1 if concepts have a single name, 3 or more for ambiguous data.
  
  \item{ConceptClassRestriction (optional)}\\
  Instead of mapping all concepts of the same concept class for all concept classes, 
  the mapping will be restricted to all concepts of the concept class(es) specified here.
  
  \item{DataSourceRestriction (optional)}\\
  Instead of mapping across all data sources, the mapping will be restricted to the data source(s) specified here.
  
  \item{WithinCVMapping (optional)}\\
  Instead of mapping across all data sources, the mapping will be performed within the data soure specified here.
  
  \item{ExactNameMapping (optional)}\\
  Forces exact matching.

\end{itemize}
    
    
%%%%%%%%%%%%%%%%%%%%%%%%%%%%%%%%%%%%%%%%%%%%%%%%%%%%%%%%%%%%%%%%%%%%%%%%%%%%%%%%%%    
\section{Sequence based Ortholog prediction}
Predicts orthologs between species based on a best bidirectional hits.

\begin{itemize}
  
  \item{SequenceType (required)}\\
  Sequence type must inherit from either Amino Acid (AA) or Nucleic Acid (NA).
  
  \item{EValue (required)}\\
  Cutoff for the BLAST E-Value.
  
  \item{Cutoff (optional)}\\
  The minimum length of sequence to allow as a valid alignment.
  
  \item{ScoreCutoff (required)}\\
  Score above which to register similar sequences.
  
  \item{ProgramDir (required)}\\
  The directory where the sequence alignment program is located.
  
  \item{TaxId (optional)}\\
  Taxonomy identifiers to include in the paralog prediction.
  
  \item{Overlap (optional)}\\
  The minimum overlap to tolerate
  
  \item{SequenceAlignmentProgram (required)}\\
  The Aligment program to use ({\it{e.g.}}, blast/patternhunter/fsablast).

\end{itemize}
    
    
%%%%%%%%%%%%%%%%%%%%%%%%%%%%%%%%%%%%%%%%%%%%%%%%%%%%%%%%%%%%%%%%%%%%%%%%%%%%%%%%%%    
\section{Sequence2pfam}
The sequence to Pfam mapping method maps a protein with an attached amino acid sequence to Pfam protein-family entries.  

\begin{itemize}
  
  \item{ProgramDir (required)}\\
  Specify blast/hmmer directory.
  
  \item{PfamPath (required)}\\
  Location of the Pfam database.
  
  \item{TmpDir (required)}\\
  Temporary directory.
  
  \item{Evalue (required)}\\
  E-value cutoff argument.
  
  \item{Method (required)}\\
  BLAST (default), Hmmer or Decypher.
  
  \item{BitScore (optional)}\\
  Bitscore cutoff argument. (NB: will only work with decypher).
  
  \item{IgnorePfamAccessions (optional)}\\
  If true, the mapping method tries to identify the protein family even if there are already protein family annotations added to the protein.
  
  \item{HMMThresholds (optional)}\\
  The HMM THRESHOLDS (can be null) specifies the use of one or more of the threshold values specified in a hidden Markov model as a minimum criteria 
  that search results must meet to be presented in the output file. 
  Pfam and other curated databases of hidden Markov models may contain within them annotation lines specifying per-model threshold values 
  for scoring alignments that use the model ({\it{i.e.}}, GA, NC or TC).
  
  \item{ConceptClass (required)}\\
  The ConceptClass to align to Pfam domains
  
  \item{AttributeName (required)}\\
  The AttributeName containing sequences to align to Pfam domains

\end{itemize}

    
%%%%%%%%%%%%%%%%%%%%%%%%%%%%%%%%%%%%%%%%%%%%%%%%%%%%%%%%%%%%%%%%%%%%%%%%%%%%%%%%%%    
\section{StructAlign based mapping}
Implements the StructAlign mapping which maps two concepts if they have the same structure of neighbours at specified depth.

\begin{itemize}
  
  \item{EquivalentConceptClass (optional)}\\
  This option should contain a pair of concept classes seperated by a comma ({\it{e.g.}}, ``Thing,EC'').
  The usage of this setting is to allow the mapping method to cross the concept class boundary in some special cases
  and thus be able to map for example similar GO and EC concepts to each other.
  
  \item{AttributeRestriction (optional)}\\
  This will limit the mapping method to only map concepts when the attribute value with the attribute name
  specified by this parameter is the same.
  
  \item{ExactSynonyms (optional)}\\
  Force matching of exact synonyms (preferred concept names) only.
  
  \item{Depth (required)}\\
  Depth of graph traversal to look for other matches.
  
  \item{ConceptClassRestriction (optional)}\\
  Instead of mapping all concepts of the same concept class for all concept classes, 
  the mapping will be restricted to all concepts of the concept class(es) specified here.
  
  \item{DataSourceRestriction (optional)}\\
  Instead of mapping across all data sources, the mapping will be restricted to the data source(s) specified here.

\end{itemize}


%%%%%%%%%%%%%%%%%%%%%%%%%%%%%%%%%%%%%%%%%%%%%%%%%%%%%%%%%%%%%%%%%%%%%%%%%%%%%%%%%%    
\section{Text Mining based mapping}
The text-mining based mapping method links concepts of any given Concept Class to concepts of type Publication
if one or more of their names occur(s) in either the title or the abstract of the publication (both saved as attributes).
The Lucene environment is used to search for concept names (terminologies, controlled vocabularies) in title and abstracts (attributes).
Text-mining based mapping can be used to create relations of type ``is\_r'' between concepts of class
{\it{e.g.}} Gene Ontology (GO), EC, Gene, Protein, Taxonomy, {\it{etc.}} to concepts of class Publication.
Evidence sentences supporting the mapping are attached to the relation
as well as a text-mining score (TF-IDF\footnote{\url{http://nlp.stanford.edu/IR-book/html/htmledition/tf-idf-weighting-1.html}}).
This score is actually boosted if a word is found in the title rather than in the abstract of the article.

\begin{itemize}
  \item{ConceptClass (required)}\\
  The Concept Class that will be mapped to concepts of type Publication.
  Multiple instances of this parameter are allowed.
  
  \item{OnlyPreferredNames (optional)}\\
  Set to true to consider preferred concept names of concepts only.
  
  \item{UseFullText (optional)}\\
  Set to true to search through full text articles if available.
  
  \item{Search (required)}\\
  Choose one instance between the following: Exact (default value), Fuzzy, Proximity, And.
  Exact only looks for exact occurrences of the words in the title and the abstract of the publication.
  Fuzzy looks for exact occurrences as well as similar words to the query and it allows any order of words.
  Proximity looks for words within the given distance (10 words by default).
  And operates in the same way as Exact except it accepts any order of words.
  
  \item{Filter (optional)}\\
  In many cases, a publication is mapped to many concepts of different concept classes.
  This optional parameter allows users to eliminate some of the low level mappings.
  Choose one instance between the following: LowScore, MaxSpecificity, BestHits.
  It is also possible to combine filters (comma separated).
  Note the order of the filters then has an importance.
  LowScore removes all hits with a TF-IDF score lower than 0.3.
  MaxSpecificity only keeps the most specific hit if several hits exist within an ontology.
  BestHits only keeps the best six hits between a publication and a concept class.

\end{itemize}
