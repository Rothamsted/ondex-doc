\chapter{FAQs}
\label{cha:faqs}
%%%%%%%%%%%%%%%%%%%%%%%%%%%%%%%%%%%%%%%%%%%%%%%%%%%%%%%%%%%%%%%%%%%%%%%%%%%%%%%%%%

\begin{itemize}
\item How do I install Ondex?
See Section \ref{sec:ref_install}.

\item How do I know what a toolbar icon does?\\
Place your mouse over it and a tooltip will pop up. Alternatively, read Section \ref{sec:ref_icons}.

\item How do I know what an item of the menu does?\\
Place your mouse over the corresponding question mark and a tooltip will pop up. 
Click on the question mark to open the Ondex documentation (also available by pressing F1) which contains examples along with screenshots.

\item Can I search for a specific element in the network and get Ondex to zoom on it?\\
Yes, see Section \ref{sec:search}.

\item Is there an import wizard for tab-delimited data files?\\
Yes. When there is not a parser available to upload data into Ondex, this general tab-delimited parser can help. 
You will need to tell Ondex which column corresponds to what kind of data. 
The simplest way to use it at the moment is through the scripting interface offered in the ``Console'' (Tools -$>$ Console). 
See Section \ref{sec:parsing_tabdel} for more information.

\item What is the largest graph I can load?\\
For the GUI to respond within reasonable time, you can load up to 100,000 elements (concepts + relations). 
For such large graphs, the visualisation will be empty at the beginning. 
For a responsive visualisation please make sure not to show more than 5,000 elements (concepts + relations) at any one time.

\item I keep getting a Java heap error. What can I do?\\
You can change the amount of memory you give Ondex when launching it. 
In order to do so, change the number after the Xmx option in the command line contained in the runme script (.bat for Windows, .sh for Linux)
or update the corresponding line in the ovtk2.l4j.ini file for Windows. 
Be careful not to set this number any higher than your computer's capabilities (on 32-bit Windows it is 1200M).

\item Where can I find data files?\\
If you have downloaded the tutorial zip file, you can find data files in the tutorial directory's subfolders.
Other data files are available on \url{http://www.ondex.org/doc.html}.

\item Where can I submit a bug or a feature request?\\
\url{http://ondex.rothamsted.ac.uk} and click on``Report bugs'' or ``Request new features'' (registration/login is required).

\item How can I define new concept attributes and colour concepts based on those?\\
Right-click on a concept, select ``Show / Edit Properties'', ``View/Edit Concept attributes'', ``Add a new attribute'' (at the bottom of the window),
a ``New'' tab is created. The ``Advanced Information'' is showing and the required fields (in orange) need to be filled in. At this stage, users can either:
enter a name of their choice in ``ID Symbol'' and ``java.lang.String'' in Class for a text attribute 
(``java.lang.Integer'' for a number, ``java.lang.Double'' for a double precision number).
Click on the ``Select Attribute Name'' and select an attribute name that has been used in the past. 
Note that if you have previously entered an attribute name, you will be able to select it in this list.
Enter the value for this new attribute in the editor pane provided.
To colour concepts to show the added annotation, use the ``Colour Concepts by General Attribute'' annotator (in the ``Tools'' -$>$ ``Annotators'' menu) 
selecting the new attribute.

\item How can I avoid waiting for a big graph to be visualised?\\
You can apply filters before opening the main network. 
This will make the visual graph smaller which means it will get painted quicker.
See F1 help for a documentation on filters or Section \ref{sec:analysing} for information on various ways of filtering out data before opening the main visualisation window.


%%%%%%%%%%%%%%%%%%%%%%%%%%%%%%%%%%%%%%%%%%%%%%%%%%%%%%%%%%%%%%%%%%%%%%%%%%%%%%%%%%
\end{itemize}
