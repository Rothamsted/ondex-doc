\documentclass[a4paper,10pt]{article}

\usepackage{xspace}
\usepackage{url}

%% definitions for the term
\newcommand{\defn}[1]{\begin{itemize}\item\textbf{Definition: }#1\end{itemize}}
%% examples of how terms might be used. 
\newcommand{\example}[1]{\begin{itemize}\item\textbf{Example: }#1\xspace\end{itemize}}
%% notes which qualify the definitions
\newcommand{\note}[1]{\begin{itemize}\item\textbf{Note: }#1\end{itemize}}
%% Suggestions for things we might change
\newcommand{\suggest}[1]{\begin{itemize}\item\textbf{Suggest: }#1\end{itemize}}
%% individual fields
\newcommand{\field}[1]{\textit{#1}\xspace}
\newcommand{\term}[1]{\texttt{#1}\xspace}
\newcommand{\todo}[1]{\begin{itemize}\textbf{TODO:} #1\end{itemize}}

\newcommand{\cc}{\term{ConceptClass}}
\newcommand{\co}{\term{Concept}}
\newcommand{\ca}{\term{ConceptAccession}}
\newcommand{\cn}{\term{ConceptName}}
%% \newcommand{\newline}{\vskip 0.5cm \noindent}

\title{Definitions for the Core terms within ONDEX: Concept}

\begin{document}
\maketitle

\section{Introduction}

This document is intended to describe the usage of the terms \co, \ca, and \cn  within Ondex. It describes the current usage, proposes a normative usage (How to use it) and suggestions for further Ondex development (Developer recommendation). 


\section{Description and current usage}

\subsection{\co}
\defn{A \co is a node that represents a bioinformatics or biological entity.}
\example{A Gene Ontology.}
\example{A Uniprot entry.}
\example{A specific drug.}
\example{A PubMed article.}


The elements of a \co are as follows:
\begin{itemize}
\item \field{id}
\item \term{CV}
\item \term{ConceptClass}
\item \term{ConceptNames}
\item \term{ConceptAccessions}
\item \term{Evidence}
\item \field{parser id}
\item \field{description}
\item \field{annotation}
\item \term{GDS}
\end{itemize}

\subsection{\field{id}}
Each \co has a numeric identifier that is automatically generated by the system. The scope of this identifier is the graph: there can only be one concept with the same identifier in a graph, but two concepts with the same identifiers in two distinct graphs are not necessarily the same.

\subsection{\term{CV}}
Each \co has exactly 1 \term{CV} designating source, for more discussion on this, see the \term{CV} document.

\subsection{\cc}
Each \co has exactly 1 \cc designating ``type'', for more discussion on this, see the \cc document.

\subsection{\cn}
\defn{A \cn is a $\langle$ \field{name}, \field{preferred} $\rangle$ pair, where the \field{name} is a ``label'' for a \co, and the boolean value \field{preferred} indicates whether the label is somehow ``preferred''.

All the \cn s associated to the same \co are supposed to be synonymous.

``preferred'' \cn s seems to be used for display purposes only\footnote{Please comment on this}.
}

\subsection{\ca}
\defn{A \ca is a triple $\langle$ \field{accession}, \term{CV}, \field{ambi} $\rangle$ where \field{accession} is an identifier, \term{CV} acts as a namespace, (further details of \term{CV} are in the \term{CV}document), and \field{ambi} denotes whether these identifiers uniquely identify the concept (see \term{CV} document for further discussions).}

\subsection{\term{Evidence}}
Every \co must be supported by some evidence, for further discussions on this, see the \term{Evidence} document.

\subsection{\field{parser id}}
The canonical description found for this was, ``This can be an alternative textual identifier for the concept which is more understandable.''


\subsection{\field{description} and \field{annotation}}
The canonical description found was ``Textual information about a concept is contained in annotation and description.''

\subsection{GDS}
Arbitrary annotations can be associated with a \co through the \term{GDS}, for more discussion on this, see the \term{GDS} document.

\section{Observations}
Most of the observations about \co are related to its associated metadata elements and can be found in the respective documentation. A few minor observations are the following:
\begin{itemize}
\item{Some ``fields'' are mandatory and some are not. However, in their usage, some of the mandatory ones often have null or default values (e.g.: EvidenceType IMPD in Evidence, or an empty string in ParserID). It seems that not all mandatory fields are necessary to characterize a \co}

\item{There is some slight dis-homogeneity between ``annotations'' (Name, Description...) in the metadata and in \co. In the metadata Ondex provides Description, ID and Name, while in \co we have Annotation, Description, zero or more names.}

\item{There is some slight dis-homogeneity in the definition of Ondex identifiers. They are numbers in \co, and text in the metadata. This indirectly implies that concept identifiers  exist on per-graph basis, while metadata identifiers are common to all graphs (this is reasonable, see document on scopes of information for more details).}

\item{There can be zero or more names associated to a \co, that are intended to be synonyms, and zero or more preferred names. It is not possible to specify more precise notions of synonymy (hyponymy hypernymy), that are sometimes found in biomedical information resources. More than one preferred name can be present, which weakens the notion of ``preferred name'' (though this remains a suitable notion) }

\item{Parser ID is of unclear usage. It seems that it can be used to provide an ID that is not only existing on a per-graph basis (as the Ondex ID). Accession can be used as well for this but in a less immediate way. }

\item{\co can have one and only one class. However, there can be concepts belonging to more classes or to no classes (something yet to be classified\footnote{A concept without a known class could be associated to the most generic type ``Thing'', if this association can be changed when more information is present.}). Concepts belonging to more classes are particularly possible where the same concept can be classified in different ways in different contexts. For instance a lipid can be classified as a molecule in a biomedical ontology, or perhaps as food in a hypothetical diet ontology. In the usage of Ondex as a knowledge base, in general one \co can be associated to more classes. (see the document on scope of information in Ondex for further discussion)  }
\end{itemize}


\section{Recommended usage}
Most of the instructions on how to use the metadata elements associated to \co can be found in the relative documents. A relevant issue is deciding what is a concept, what should be in GDS, and what in the metadata layer. This is discussed in a distinct document.
\begin{itemize}
\item {parser\_id } should not be used \footnote{this require some more discussion, please comment on this. One possible usage (not from the developers, but embedded in the APIs) could be to store URIs here.}
\item{Names} should report labels that are found to be used to identify a concept. These should be normalized (singular, first letter, names and acronyms in capital). The indication of preferred names is arbitrary and should not be considered as having a computable meaning (the preferred name should be the most common name, if clearly identifiable, and it should be in English, UK spelling). Anything that includes a verb cannot be considered as a name. For instance "apoptosis inducing protein" is a name, while "protein that induces apoptosis" is not (it is a description).
\item{Description includes an extensive characterization of \co that is not intended to be interpreted automatically. Within description, some convention can be demanded as a workaround for limitations in the Ondex data structure. This is discussed in other documents where appropriate.}
\item{Annotation, as described elsewhere, should not be used.}
\end{itemize}
\section{Future development}



Most of the relevant future developments pertain to elements of the meta-data layer that are associated to concepts. For concepts, it is worth to eliminate fields that are redundant, such as ``Annotation''.
\end{document}

