\documentclass[a4paper,10pt]{article}

\usepackage{xspace}
\usepackage{url}

%% definitions for the term
\newcommand{\defn}[1]{\item\textbf{Definition: }#1\xspace}
%% examples of how terms might be used. 
\newcommand{\example}[1]{\item\textbf{Example: }#1\xspace}
%% notes which qualify the definitions
\newcommand{\note}[1]{\item\textbf{Note: }#1\xspace}
%% Suggestions for things we might change
\newcommand{\suggest}[1]{\item\textbf{Suggest: }#1\xspace}
%% individual fields
\newcommand{\field}[1]{\textit{#1}\xspace}
\newcommand{\term}[1]{\texttt{#1}\xspace}
\newcommand{\todo}[1]{\textbf{TODO:} #1\xspace}


\title{Definitions for the Core terms within ONDEX: Preliminary}

\begin{document}
\maketitle

\section{Introduction}

This document is intended to describe the usage of the term \term{CV} within
Ondex. It describes the current usage, proposes a normative usage (How to use
it) and suggestions for further Ondex development (Developer recommendation). 


\section{CV}

\subsection{Current usage}

\begin{itemize}

  \defn{Describes the bioinformatics origin of \term{Concept} or
    \field{accessions} of \term{Concept}{}s.}
  
  \example{``UNIGENE'', ``GO'', ``unknown'', ``AFFYMETRIX'', ``BROAD'', ``NWB''}
\end{itemize}

\subsection{Observations}

\newcommand{\cv}{\term{CV}}
The \term{CV} is currently used with several different meanings. 

\begin{itemize}

\item \term{CV} is often used to indicate a namespace: go concepts with ``GO''
  \term{CV}s are used in many parsers. This use as a namespace is normal, when

  \term{CV} is used in a \term{ConceptAccession}; here it defines the set of
  things within which the \field{accession} is valid.
  
\item \term{CV} can indicate a format from which these data where parsed
  (e.g: ``NWB'').

\item \term{CV} can indicate a database from which information is
  derived (e.g.: ``ATRegNet'', defined on a ``per parser'' basis), hence
  representing some provenance information.
  
\item \term{CV} is at times used as a graph identifier in operations that need to distinguish between different graphs.

\end{itemize}

We can draw some conclusions form CV. 

\vskip 0.5cm
where the \cv
functions as a namespace, and for an unambiguous \term{ConceptAccession}, this
seems to imply that two concepts with the same \field{accession} and the same
\term{CV} are equivalent. However, this is not always true; for instance for
GeneDB, \term{CV} + species + \field{accession} would be required to specify a
``key''
\footnote{Most of the mapping methods have an argument "GDSEqualsConstraint" which compares the GDS attribute representing the species (e.g. in TAXID). This compensate to some extent for the lack of precision in the definition of \term{CV}, but fail in providing an explicit representation of identity } .


It is to be verified if this role of \term{CV} is consistent with the usage of the
\field{ambiguous} flag.
\vskip 0.5cm

Some \term{CV}s such as ``KAZUSA'' and ``HUGE'' are not independent: ``HUGE''
refers to a subset of ``KAZUSA''. 
\vskip 0.5cm

\newcommand{\relation}{\term{relation}} \term{CV} and \term{EvidenceType} seem
to be closely related to each other; the difference between the two is not
clear
\footnote{
In the intended usage, a CV should answer the question "where is something coming from", an evidence type should capture "how something has been created". See this document and the document on  \term{EvidenceType} and \term{Evidence} for an precise analysis.
}.
\vskip 0.5cm

It is not clear why \term{CV} and \term{EvidenceType} can be applied to some element of the Ondex graph and not to others\footnote{
In the current usage of Ondex, an accession is part of the Concept. Therefore the EvidenceType on a Concepts holds for an accession. If something on a Concept gets changed, then usually its EvidenceType should change too. (Not every plugin does it, because of laziness). See the documents on \term{EvidenceType} and \term{Evidence} for an analysis of current usage and normative usage of \term{EvidenceType}
 }.

For example, \field{accession} can be qualified with a \cv, but not with a \term{EvidenceType}.
A \term{Relation} can have an \term{EvidenceType}, but not a \cv. It's not clear why this is different than for \term{Concept} \footnote{A relation has an implicit CV constituted by the concepts taking part in the relation with the following interpretation: if (fromConcept.CV == toConcept.CV) relation.CV = fromConcept.CV
else
relation.CV = fromConcept.CV + toConcept.CV. This implicit attribution of \term{CV} to a relation has two problems: it is implicit (should be declared) and it is not, in general, appropriate. What if a relation comes from a third CV, respect to the elements that participate in it?)
}. 



\footnote{
In the current usage of Ondex, an accession is part of the Concept. Therefore the EvidenceType on a Concepts holds for an accession. If something on a Concept gets changed, then usually its EvidenceType should change too. (Not every plugin does it, because of laziness). See the documents on \term{EvidenceType} and \term{Evidence} for an analysis of current usage and normative usage of \term{EvidenceType}
 }
 
 


\subsection{How to use it}

It is important to distinguish between the different meanings of \term{CV}: namespace,
provenance and so on. 

\subsubsection{Usage of CV in \term{ConceptAccession}}

When \cv is attached to \term{ConceptAccession}, it is used to describe
identity; \cv is being used as a namespace, within which the
\field{acccession} has a specific meaning. 

For example, when using OBO ontologies as a \term{CV} in
\term{ConceptAccession}, the same \field{accession} (e.g. 00001) can refer to
different terms in different ontologies. \term{CV} in this case should specify
the ontology (the scope) in which the accession is applicable (e.g.
\underline{GO}:000001, \underline{TR}:000001, etc). In this context, OBO would
not be an appropriate \term{CV} as OBO:000001 may refer to terms in more than
one ontology.

\term{CV} should not refer to the technology used to generate data (e.g:
Affymetrix), nor to the institute providing the data (e.g. Broad), but should
refer to knowledge bases (e.g. UniProt) or other ``resources'' (e.g. GO) that
provide the full definition for that term; a pair
(\term{CV},\field{accession}) as it appears in \term{ConceptAccession} should
be semantically equivalent to a URI for a concept.

It is often the case, that a namespace is equivalent to an authority; so, for
instance, Uniprot operates as both a namespace and an authority. However, this
is not always the case; it is the namespace that should be expressed where
they differ. For example, in the case of KEGG, a \term{CV} like ``KEGG''
doesn't correspond to namespace, as it could be about ``KEGG GENES'', ``KEGG
PATHWAYS'', etc. So, KEGG PATHWAYS should be used, rather than KEGG. 

For \term{concept}s derived from resources that provide global
identifiers, there should always be at least one unambiguous \field{accession}
within ONDEX. For instance, concepts representing ontology terms should always
be annotated with an unambiguous \term{ConceptAccession} specifying namespace
(e.g. GO) and \field{accession} (e.g. 00001).

To support the evolution of Ondex into a knowledge base, a \term{CV} may be needed to be used to identify ``local entities''  in a
\term{ConceptAccession} (``local entities'' are entities whose identity is relevant on a per-task or per-user basis only). This is not possible in Ondex at the moment
\footnote{There is a not so common usage of the CV "VO" (Virtual Ontology) for exactly this local data only created and held within the ONDEX system. This only partially address the need for the identification of ``local'' information.}
. 

\subsubsection{Usage of \term{CV} in \term{Concept}}


When \term{CV} is used for a \term{Concept}, it relates to proof and the
meaning to which it refers is the one of provenance, intended as an aspect of
evidence.

In the case of information originating from a database, \term{CV} (provenance)
is the original source of data, intended as the most specific authority that
is responsible for the validity of the data (this is often the last data
source from which this concept is derived, see examples).

Other information related to ``proof'' will be dealt with in \term{Evidence}.

Terms like ``unknown'' should be avoided in a \term{CV}. For a \term{CV}
referring to a \term{Concept} for which an authority is not relevant, for
instance, a new \term{Concept} introduced during an analysis (e.g. clusters
created via K-means), the term ``NotApplicable'' should be used (Note this is distinct from the need of a local namespace previously introduced).

For example, \term{Concept}s representing UniProt proteins imported from the
TAIR database should be associated with the \term{CV} TAIR rather than
\term{CV} UniProtKB. This is not to be confused with the usage \term{CV} in
\term{ConceptAccession} which in this case, would be \term{CV} UniProtKB.


\subsubsection{Other usage of \term{CV}}

Any usage other than the ones defined above should be forbidden\footnote{By ``forbidden'' we mean that this would break the compliance with this documentation} in \term{CV}.
\term{GDS} can be used for these purposes if they are required.




\subsubsection{Adding new \term{CV} terms - Intructions for Ondex developers}

When a \term{CV} term is needed, it should be first assessed whether a
suitable \term{CV} is already defined. At the moment, this can be done by
inspecting the ondex metadata file (the Ondex suite provides a graphical tool for this).

Although technically possible in Ondex, new \cv{}s should not be added to the
metadata when writing software. If new \term{CV} term is needed, these should
be introduced to the official metadata file first. In order to do this (at the
moment), it is necessary to write an email to: \url{metadata@ondex.org}, with
a brief description (and hyperlink) of the resources for which a \term{CV} is
required, and its intended usage.


\subsubsection{Adding new \term{CV} terms - Instructions for metadata editors}

At this stage, \term{CV}s can be annotated in Ondex with
\field{id}, \field{fullname}, and \field{description}. \field{id}s should
be unique.

\begin{itemize}
\item For \term{CV}s used in \term{ConceptAccession}, 
  
  \begin{description}
  \item[\field{id} should be:] the prefix ``NS-'' followed by the common
    abbreviation, if there is one (e.g. GO, IEV, etc), the name of the
    resource starting in capital letters and in Camel Back notation
    otherwise. No spaces should be included and acronyms are acceptable
    (e.g. UniProtKB, BioCyc, etc).
  \item[\field{fullname} should be:] the name of the resource without
    acronyms or abbreviations (e.g. Gene Ontology for GO, etc)
  \item[\field{description}:] the content of this field should be structured
    as follows: 
    \begin{description}
    \item[NAMESPACE:] a namespace corresponding to this resource. The format
      of this namespace is defined in a separate document. 
    \item[URL:] the url of the resource. 
    \item[Text:] A natural language description (20-100 words) of the
      resource.
    \end{description}
  \end{description}

\item for CVs used in Concept
  \begin{description}
  \item[\field{id} should be:] the prefix "SRC-" folowed by the common
    abbreviation, if there is one (e.g.: GO, IEV, etc), the name of the
    resource starting in capital letters and in Camel Back notation
    otherwise. No spaces should be included and acronyms are acceptable
    (e.g. UniProtKB, BioCyc, etc).
  \item[\field{fullname} should be:] the name of the resource without
    acronyms or abbreviations (e.g. Gene Ontology for GO, etc)
  \item[\field{description}:] the content of this field should be structured as follows: 
    \begin{description}
    \item[URL:] the url of the resource 
    \item[Text:] A natural language description (20-100 words) of the resource.
    \end{description}
  \end{description}
\end{itemize}

  % \example full example to be added here


\subsection{Developer recommendations}

In this section, we suggest changes that we feel will increase the ease of
use, understandability of ONDEX; this should increase substantially to the
reusability and interoperability of ONDEX resources. 


\term{CV} should be split into ``Namespace'' and into an ``Evidence Object''.
This ``Evidence Object'' will be discussed in the ``Evidence'' document.
Correspondingly, the terms now in \term{CV} should be split in two lists (note
that the prefixes suggested in the previous section can be eliminated in this
process)

Additional properties now suggested as structured fields in the
\field{description} should be made explicit (e.g. Namespace, URL).

We have required in these recommendations that a namespace is always
associated to the use of \term{CV} in \term{ConceptAccession}s. This forces
the inclusion of all information that may be necessary to qualify the
accession in the namespace (e.g.: species, version, etc). While this design
has some limits, it is a viable solution in the medium term and a more
extensive documentation on it will be provided separately.

An additional property could be included for \field{accession}, to specify
``authorities'' generating \field{accession}s, when the scope of their
validity is not known. For instance, for a database ``PlantDB'', dealing with
a range of organisms, it can be known that \field{accession}s provided for
genes are ``PlantDB'' \field{accessions}, but it may not be known whether
these \field{accession}s are unique within ``PlantDB'', or within a specific
organism.

The system should be designed so that it is not necessary for the user to
enter terms as ``unknown'' or ``NotApplicable''.

In the long term, it should be possible to define different levels of scope
for \term{Concepts}, without the need to rely on \term{ConceptAccession} and
\term{CV} for that. This will be discussed in the documentation relative to
\term{Concept} and \term{ConceptAccesion}.

It would be beneficial to implement a directory for \term{CV} terms. Ideally,
this should be based on a collaborative environment based on standard formats,
with a bridge provided to the Ondex metadata layer. For instance the metadata
could be edited in Protege and a simple script to convert it to the Ondex
format could be implemented.

The name ``CV'' is not helpful, as many things in Ondex are controlled
vocabularies; however \term{CV} only appears to apply to one of these.


\end{document}
