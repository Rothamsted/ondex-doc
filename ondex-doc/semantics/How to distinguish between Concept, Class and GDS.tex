\documentclass[a4paper,10pt]{article}

\usepackage{xspace}
\usepackage{url}

%% definitions for the term
\newcommand{\defn}[1]{\begin{itemize}\item\textbf{Definition: }#1\end{itemize}}
%% examples of how terms might be used. 
\newcommand{\example}[1]{\begin{itemize}\item\textbf{Example: }#1\xspace\end{itemize}}
%% notes which qualify the definitions
\newcommand{\note}[1]{\begin{itemize}\item\textbf{Note: }#1\end{itemize}}
%% Suggestions for things we might change
\newcommand{\suggest}[1]{\begin{itemize}\item\textbf{Suggest: }#1\end{itemize}}
%% individual fields
\newcommand{\field}[1]{\textit{#1}\xspace}
\newcommand{\term}[1]{\texttt{#1}\xspace}
\newcommand{\todo}[1]{\begin{itemize}\textbf{TODO:} #1\end{itemize}}

\newcommand{\cc}{\term{ConceptClass}}
\newcommand{\co}{\term{Concept}}
\newcommand{\rt}{\term{RelationType}}
\newcommand{\re}{\term{Relation}}
\newcommand{\gds}{\term{GDS}}
%% \newcommand{\newline}{\vskip 0.5cm \noindent}

\title{How to distinguish between Concept, ConceptClass, GDS}

\begin{document}
\maketitle

\section{Introduction}
This document is intended to aid in deciding whether a given piece of data should be represented as \co, \cc, or \gds.

Discussions on the structure of the above are dealt with their respective documents.

\section{How to distinguish}
\co, \cc and attributes (\gds) corresponds in a large part to the distinction between individuals, classes and attributes. 
\subsection{Concept}
\begin{itemize}
\item A \co can be used to represent an individual, such as a person (the author of a publication), a patient affected by a given pathology, a single mouse treated with a stimuli. What is an individual is intuitively understood for the physical world, especially at a macroscopic scale.
\item In the realm of molecular biology and bioinformatics it is often impossible to discern individuals. This is possible for a patient or an animal, but is not possible for a molecule or gene. When we say that a P53 induces, under some circumstances, apoptosis, we are not talking about a single protein that in a specific subject (say, mouse number 007) induces a process of apoptosis of one cell. We make a general assertion on a multitude of similar proteins that, in similar circumstances, induce a similar process.
We are effectively talking about classes of proteins, processes, or more in general entities.
For these kind of entities we can use \co, but noting that we refer not to individuals, but to prototypical individuals (this can be noted in the description as described in the document on \co).
We can choose to represent in Ondex both the \co and the related  \cc (e.g.: Apoptosis prototypical \co and Apoptosis \cc), or only the \co, that in this case may be linked to a more generic \cc (e.g.: Apoptosis prototypical \co has \cc Biological Process).
\end{itemize}
\subsection{ConceptClass}
\begin{itemize}
\item \cc are used to represent groups of entities that share similar characteristics (also known, with some approximation, as classes or universals). Examples of classes or universals\footnote{The distinction between classes and universals is not relevant for the scope of this document. Intuitively, some class as ``Protein'' represent a ``relevant'' class commonly understood by people, while ``red cat food'' is a more specific class that has not a name. Class with names tends to be universals}, that can be represented as \cc, are humans, protein, biological process.  The relation between individuals and classes is ``instantiation''.
Classes are ordered in a hierarchy of abstraction (how many properties the individuals they denote have in common). E.g: Apoptosis (programmed cellular process and results in cell death) is a less general class than cellular death (results in cell death). The most general class is called, in some context, ``Thing'' (no restrictions).
\item Ontologies,by their own definition, should be represented as \cc. They may as well be represented as \co, intended as a protypical instance of that class. However, unlike individuals, they should always be represented also as classes. We note in fact that the relation of ``abstraction" (is\_A) makes sense only among classes and hence it should be represented only in the metadata. Between two ontology term represented as \co, ``is\_A'' cannot be asserted (they have the meaning of protoypical instances, not of classes).
\item Ontology term and other ``bioinformatic entities'' for instance, database identifiers, can also be seen as individuals of an ``annotation type''.  For instance GO:0002260  is an individual ontology term, that refers to a class of processes (leukocytes homeostasis). 
When GO:0002260 is represented as a \co, instance of \cc GO:0002260, it represents a prototypical process, while if used as instance of \cc "ontology term" it represents only an annotation. Unless a specific application of Ondex is concerned with the analysis of ontologies per se (not what they represent), ontology terms should always be represented as \cc, eventually with an associated \co. 
\end{itemize}
\subsection{Attributes}
\begin{itemize}
\item Attributes in \gds should be used to represent ``dependent entities'', usually qualities. For instance a mass is not something that can exist independently from a protein or molecule, therefore it should be represented as an attribute in a \gds. Other examples are protein structure, sequence of a gene or protein. 
\item It is worth noting that we often use the same terms to refer to information entities and what their refer to. For instance a molecular structure, intended as what it refers to, is an attribute of a prototypical molecule (or a class of molecules), while intended as a string describing a physical structure, it is an individual on its own. Thinking about properties of entities helps in making distinction: when you say that you can represent a physical structure in a given format, you refer to the information artifact, while when you say that a process changes the physical structure of a protein, you refer to the attribute.

\end{itemize}



% Two examples from GDS document are mass and genomic location

% two examples from GDS document are tax id and phenotype being co/re



\end{document}