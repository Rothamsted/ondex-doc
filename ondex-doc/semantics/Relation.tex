\documentclass[a4paper,10pt]{article}

\usepackage{xspace}
\usepackage{url}

%% definitions for the term
\newcommand{\defn}[1]{\begin{itemize}\item\textbf{Definition: }#1\end{itemize}}
%% examples of how terms might be used. 
\newcommand{\example}[1]{\begin{itemize}\item\textbf{Example: }#1\xspace\end{itemize}}
%% notes which qualify the definitions
\newcommand{\note}[1]{\begin{itemize}\item\textbf{Note: }#1\end{itemize}}
%% Suggestions for things we might change
\newcommand{\suggest}[1]{\begin{itemize}\item\textbf{Suggest: }#1\end{itemize}}
%% individual fields
\newcommand{\field}[1]{\textit{#1}\xspace}
\newcommand{\term}[1]{\texttt{#1}\xspace}
\newcommand{\todo}[1]{\begin{itemize}\textbf{TODO:} #1\end{itemize}}

\newcommand{\cc}{\term{ConceptClass}}
\newcommand{\co}{\term{Concept}}
\newcommand{\rt}{\term{RelationType}}
\newcommand{\re}{\term{Relation}}
%% \newcommand{\newline}{\vskip 0.5cm \noindent}

\title{Definitions for the Core terms within ONDEX: Relation}

\begin{document}
\maketitle

\section{Introduction}

This document is intended to describe the usage of the terms \re within Ondex. It describes the current usage, proposes a normative usage (How to use it) and suggestions for further Ondex development (Developer recommendation). 


\section{Description and current usage}


\defn{A \re is a connection between one, two, or three concepts.}
\example{A prototypical mRNA and a prototypical protein may be connected because said protein is a translation of said mRNA.}
\example{A prototypical gene and a prototypical disease phenotype are connected because the gene is causative for the disease phenotype. This is supported by a publication which is used to qualify this relationship. (third concept)}
\example{``is a'' is a relationship that would apply between a concept and itself.}


The elements of a \re are as follows:
\begin{itemize}
\item \field{id}
\item \rt
\item \term{Evidence}
\item \term{GDS}
\item \field{qualifier}
\end{itemize}

\subsection{\field{id}}
Each \re has a numeric identifier that is automatically generated by the system. The scope of this identifier is the graph: there can only be one \re with the same identifier in a graph, but two \re with the same identifiers in two distinct graphs are not necessarily the same.

\subsection{\rt}
Each \re has exactly 1 \rt designating ``type'', for more discussion on this, see the \rt document.

\subsection{\term{Evidence}}
Every \re must be supported by some evidence, for further discussions on this, see the \term{Evidence} document.

\subsection{GDS}
Arbitrary annotations can be associated with a \co through the \term{GDS}, for more discussion on this, see the \term{GDS} document.

\subsection{\field{Qualifier}}
Each \re may have a qualifier, this provides the ability to have a third node in relation. This tends to be used, thus far, for holding publications to evidence relations.


\section{Observations}

Most of the observations about \re are related to its associated metadata elements and can be found in the respective documentation. A few minor observations are the following:


\begin{itemize}

\item There is a restriction of only one \re between any two concepts of the same \rt. As \rt s can be fairly general, this disallows some things to be expressed. For example, between gene A and B. If tblastx, blastn, blat, and Smith-Waterman alignment were conducted between them, the \rt would be ``h\_s\_s'' (has similar sequence), but only one set of bit-score / e-values could be stored, as one could only set up 1 \re between A and B.

\item It is possible to have multiple ternary \re s with the same \rt where the two primary nodes are the same so long as the qualifier is different.

\item Unlike \co , there is no default place for holding a human readable description for a \re.

\item It's not possible to assert where a \re is from. i.e. there is no \term{CV}. Hypothetically, a \term{CV} may be derivable from looking at the \term{CV}s of the \co s that it attaches.

\item Like \co , there is a difference between \field{id}s in \co and \re and those in metadata elements, comments about this are made in the \co document.

\item There is no multi-classing allowed for \re , each \re may only have one \rt . It is not clear whether this would cause problems further down the line.

\end{itemize}


\section{Recommended usage}
Most of the instructions on how to use the metadata elements associated to \re can be found in the relative documents. A relevant issue is deciding what is a concept, what should be in GDS, and what in the metadata layer. This is discussed in a distinct document.

\section{Future development}
There are not future developments suggested at the time of this writing.


\end{document}

