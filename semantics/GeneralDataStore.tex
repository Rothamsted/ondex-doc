\documentclass[a4paper,10pt]{article}

\usepackage{xspace}
\usepackage{url}

%% definitions for the term
\newcommand{\defn}[1]{\item\textbf{Definition: }#1\xspace}
%% examples of how terms might be used. 
\newcommand{\example}[1]{\item\textbf{Example: }#1\xspace}
%% notes which qualify the definitions
\newcommand{\note}[1]{\item\textbf{Note: }#1\xspace}
%% Suggestions for things we might change
\newcommand{\suggest}[1]{\item\textbf{Suggest: }#1\xspace}
%% individual fields
\newcommand{\field}[1]{\textit{#1}\xspace}
\newcommand{\term}[1]{\texttt{#1}\xspace}
\newcommand{\todo}[1]{\textbf{TODO:} #1\xspace}

\newcommand{\an}{\term{AttributeName}}
\newcommand{\gds}{\term{GDS}}
\newcommand{\unit}{\term{Unit}}

\title{Definitions for the Core terms within ONDEX: GDS}

\begin{document}
\maketitle

\section{Introduction}

This document is intended to describe the usage of the terms \term{GDS}, \term{Unit}, and \term{AttributeName} within Ondex. It describes the current usage, proposes a normative usage (How to use it) and suggestions for further Ondex development (Developer recommendation). 


\section{Current usage}

\subsection{GDS}
A \gds or \term{General Data Store} is a set of $\langle$\an , value$\rangle$ pairs attached to Ondex entities (i.e. \term{concept}s or \term{relation}s). \gds further has a restriction that for each \an, there can only be at most one value per \an for each Ondex entity.

\subsection{\an}
\an identifies attributes of Ondex entities.
\vskip 0.5cm
\noindent
\an sets out a restriction on the set of values allowed. i.e. The \an defines the type (Integer, String, ...) that the associated value belongs to. The restriction is expressed as a Java class. \an may specify any Java class.
\vskip 0.5cm
\noindent
\an \textbf{must} have a data type associated to them, and \textbf{may} have a \unit.


\subsection{\unit}
\unit is a descriptor associated with an \an to specify what standard measurement the value of the \gds is expressed in. i.e. metre, kg.

\subsection{Examples}
Genomic location
\begin{itemize}
\item AttributeName: Genomic Location
\item Unit: kDa
\item Value: 5
\item Data type: java.lang.Double
\end{itemize}

Author
\begin{itemize}
\item AttributeName: Author
\item Unit: None
\item Value: Bloggs, J.
\item Data type: java.lang.String
\end{itemize}

\subsection{Observations}
The usage of \gds sometimes overlaps with the use of the concepts and relations, as well as the rest of the metadata.
\vskip 0.5cm
\noindent
As examples,
\begin{itemize}
\item  ``doi'' is sometimes expressed as an \an in \gds : it could be expressed as \term{conceptAccession} for a \term{Concept} of type Publication.
\item ``taxid'' is sometimes expressed as an \an in a \gds : it could be be expressed as \term{ConceptAccession}for a \term{concept} of type Species.
\item Sometimes information encoded in the \gds refer provenance information, i.e. ``BLEV'' (Blast e-values), ``Evidence Sentences''.
\end{itemize}
\noindent
\an do not distinguish between attributes that are properties of the object described, attributes that are proper of the Ondex representation and attributes that derive from analysis of larger parts of the network (they don't only depend on the entity they annotate)\footnote{this a fundamental distinction and not a question of granularity!}. Examples of the firsts are ``YearPublished'' (property of a paper) or ``Cotyledon Group''. Examples of the seconds are  ``graphical" (co-ordinates for a layout), ``text mining score'', ``link to image''. As for \an that refer to analysis depending on the whole network, two relevant examples are ``NBCONF'' (Neighborhood Confidence Value), ``ACCHITS'' (Counts of hits for accession based mappings).
\vskip 0.5cm
\noindent
As a consequence of the previous point, \gds can be used to hold both information that is amenable to be persistent, and information that is valid only on a ``per session'' basis\footnote{this unexpressed distinction is found throughout the Ondex data model, and it will be addressed in a dedicated document}. Attributes of Proteines such as weight or sequenceLocation are stable for any given state of the know-how represented in Ondex. Attributes as ``Neighborhood Confidence Value'' can change if a concept is added or substracted from Ondex.

\vskip 0.5cm
\noindent
Some \an encode some undeclared semantics. i.e. in a range value, there may a expectation of the format of this value. i.e. (value A) - (value B).
\vskip 0.5cm
\noindent
For serialising into OXL, the \gds values have a limit of 32k.

\subsection{Recommended usage}
It is important, for the interoperability of graphs generated by different ``users'', that \term{GDS}, \term{Concept} and \term{Relation} are used in an uniform way. This implies that \term{GDS} should be used only when no other way to represent the same information in Ondex are possible.
\vskip 0.5cm

\an in a \gds should only be used for information that is dependent on an entity, and that doesn't make sense by itself. All other information should be represented through other ondex features.
\vskip 0.5cm
\noindent
Examples
\begin{itemize}
\item Genomic location only make sense referring to a gene. Thus, it should be expressed as a (\an, value) pair in the \gds related to the \term{Concept} gene (in short: a GDS entry)
\item taxid and phenotype should not be expressed as a \gds entry, nor should ontology terms. In the case of taxid, it could be expressed as an accession on a species\footnote{it should be noted that this might increase the cluttering of the graph, e.g.: in AraCyc, every gene or protein will have a relation to one species. However, this may be necessary for a "base graphs" from which applications specific and simplified graphs can be derived. Alternatively, \term{Context} ``may'' be used to define species specific graphs (this needs to be more investigated, refer to the \term{Context} document).}.
\item Mass should be a \gds, mass does not make sense on its own.
\end{itemize}
\noindent
\gds should be as explicit as possible ( i.e. range values could be broken down into a min value and a max value entry).
\vskip 0.5cm
Some information as ``Authors'' of a publication concept can be represented as \term{Concept} or as a \gds, this depends on the ``commitment'' of the ondex representation: in the the case Authors are expressed through \gds, they have to be intended as an opaque annotation of a publication Concept.
\vskip 0.5cm
\noindent
There should be a distinction between permanent attributes and attributes generated on a per session basis. It is suggested that ``per session'' attributes be prefixed by the ``temp'' string. If an ``ondex repository'' would exist, only non-temp values would be stored in it.
Refer to the next session for a more articulated discussion.
\vskip 0.5cm
While GDS can have as object an arbitrary Java Object, the use of objects other than the ones corresponding to XSD datatypes is to be strictly limited for two reasons:
\begin{itemize}
\item Complex objects limit data-exchange possibilities (they most likely are not shared across systems). Hence they should be useful only when no alternatives exist.
\item Complex objects may encode some structure that should better be exposed, in its semantics, in the Ondex data model.

\item GDS are possible only for \term{Concept} and not for \term{ConceptClass}. They should apply to both, in general, as both can have attributes. 

\end{itemize}



\subsection{Future development}
The \term{gds} structure in Ondex is overloaded in its usage. It is used to address distinct tasks as: 
\begin{itemize}
\item represent attributes of the object they describe (such as mass of a protein)
\item represent values that are proper of the information object in ondex (e.g.: coordinates of a node representing a protein)
\item represent values that are relative to the attributes of the network and that don't depend only on the object they annotate (for instance, a confidence value based on the neighborhood).
\end{itemize}

Different usages of GDS result in a different lifespan, or scope, of the information they represent. We can at least distinguish between a scope that we define ``global'' that pertains to attributes proper of the object (such as mass of a protein): values in this scope would be independent versus modification of the network and from the analysis performed.
Another scope, that we can call ``local'' would include attributes such as analysis values that in general depend on a specific ``state'' of the network, and that can often be meaningful on a ``per task'' basis.
\vskip 0.5cm

In an evolution of Ondex toward a knowledge base supporting a variety of users, it is important to be able to distinguish between scopes.

At the very least it is important to distinguish between ``local'' and ``global'' scopes of attributes.  This can be done by adding a flag in the metadata layer.
\vskip 0.5cm

A more versatile solution would associate ``local'' attributes to an identifier defining the limits of their validity. This identifier would identify a ``scope'' of validity such as a specific analysis task. It could also be used to identify ``special scopes''', such as the information needed by Ondex to support a specific visualization of a network\footnote{as previously noted, the need for this distinction appears throughout Ondex, and will be dealt with in a separate document}.

In practice, this solution would be subsumed by the ability to qualify subnetworks with a named scope,  in a way that is affine to RDF ``named-graph''. This solution would be partially overlapping with the use of \term{Context} in Ondex. A more elaborated discussion on this point will be presented in the \term{Context} document.
\vskip 0.5cm
\noindent
A more advanced implementation can be provided for ``analysis'' values. Each analysis value could be associated to a signature defining its validity. For instance, for values depending on the topological features of a network, the signature could be the checksum of the ordered list of node-edge-node triples. This signature would be identical as long as the features the value depend on don't change.

\vskip 0.5cm
\noindent
\an could be characterized by Domain and Range.


\end{document}

