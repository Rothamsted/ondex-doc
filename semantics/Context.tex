\documentclass[a4paper,10pt]{article}

\usepackage{xspace}
\usepackage{url}

%% definitions for the term
\newcommand{\defn}[1]{\item\textbf{Definition: }#1\xspace}
%% examples of how terms might be used. 
\newcommand{\example}[1]{\item\textbf{Example: }#1\xspace}
%% notes which qualify the definitions
\newcommand{\note}[1]{\item\textbf{Note: }#1\xspace}
%% Suggestions for things we might change
\newcommand{\suggest}[1]{\item\textbf{Suggest: }#1\xspace}
%% individual fields
\newcommand{\field}[1]{\textit{#1}\xspace}
\newcommand{\term}[1]{\texttt{#1}\xspace}
\newcommand{\todo}[1]{\textbf{TODO:} #1\xspace}

\newcommand{\cc}{\term{ConceptClass}}
\newcommand{\co}{\term{Concept}}
\newcommand{\cxt}{\term{Context}}
\newcommand{\re}{\term{Relation}}
%% \newcommand{\newline}{\vskip 0.5cm \noindent}

\title{Definitions for the Core terms within ONDEX: Context}

\begin{document}
\maketitle

\section{Introduction}

This document is intended to describe the usage of the terms \cxt,  within Ondex. It describes the current usage, proposes a normative usage (How to use it) and suggestions for further Ondex development (Developer recommendation). 


\section{Current usage}
A \cxt is a part of an Ondex graph (sub-graph) that is associated to one concept, that ``qualifies'' this part of a graph. Examples in the current set of plugins are:

\begin{itemize}

\item Sets of genes can be grouped in contexts associated to concepts representing Chromosomes.
\item Sets of genes and relations can be grouped in contexts associated to nodes representing Pathways.
\item QTLs and publications can be grouped into contexts associated to Experiments.
\item GO terms can associated via a context to terms that are more generic.
\item QTLs can be ``the context'' for Genes (meaning that a group of genes can be grouped and associated to nodes of type QTL).
\item Proteins can be contexts, ``holding'' publication information (uniprot transformer).
\item Cultivar information can be a context (proteins associated to cultivars in grains).
\item The hierarchy encoded in EC numbers can be expressed as contexts (later numbers in the context of earlier ones)	.
\item Contexts can be used to group the results of computation (cliques, clusters) or for holding on-going computation (optimal paths transformer)
\item Pfam transformer can assign a family to itself via a context (a group of one concept is associated to itself).

\end{itemize}

\subsection{Observations}
\begin{itemize}
\item As for other elements in Ondex, there is not a distinction from \cxt representing biological groups  and \cxt representing information artifacts or results of computations. The need for this distinction is presented in the document on scope of information in Ondex.

\item It is possible to associate a concept to a context characterized by itself.

\item Contexts can be cyclical. X could be in the context for Y which is in the context of X.

\item It is technically possible to have \re in a context, without  the \co s which it connects. In general, a \cxt can contains an arbitrary subset of \co and \re in the graph, without the need to have complete `` \co \re \co'' statements. For instance if \re R connects \co C1 and \co C2. \co CXT may have \cxt containing just R, or R and C1, and so on.

\item A relevant use for contexts is filtering and visualization.

\item The use of context overlaps with other relations, such as ``part-of'' (for pathways) and ``is-A'' (for GO, EC). 

\item There seems to be two distinct uses of contexts. One use is as devices for identifying a subgraph (as when holding partial computation), and the other use is as devices to ``characterize'' a subgraph or a set of genes (as in the case of Chromosomes as context for genes). In the first case, the \cxt does not imply any particular semantic. In the second case, the \cxt add semantic information to a set of \co or \re.

\item Sometimes contexts are used to represent a ``containment'' relation. This is the case when reaction in a pathway are in a \cxt associated to a pathway, or when genes are in a  \cxt associated to a Chromosome \footnote{The definition of ``containment'' here is only intuitive.} Containment relations could be expressed as well via \re. This would be straightforward in the case of Genes and Chromosomes, apart from an increase in the complexity of the graph. For reactions and pathways, the representation of ``containment'' via relations would require the representation of reactions as \co (this is not the typical representation in Ondex, and a transformation\footnote{This transformation would be in practice a reification} would be required).

\item As \cxt s are treated as any other \co , it is possible to create \re between \cxt s. These \re s are as any other \re , and themselves can be put into \cxt s.

\item Should a \re be made with a \co that contains a \cxt , it is unclear whether this \re would refer to the \cxt or the \co . i.e. if a protein \co had publications listed as \cxt , would a \re refer to the protein or the publications.


\end{itemize}



\subsection{Recommended usage}
\begin{itemize}
\item The association of a set of concepts, relations (or in general a sub-network of an Ondex network) to a generic concept in Ondex is to be considered as an annotation intended for human readability. It carries the meaning of ``relevant for'' with no further commitment. It shouldn't be used to represent information amenable to computation, but only as a device to support filtering and visualization. For instance, in case of chromosomes and genes, the ``containment'' of genes in chromosomes should be stated via properties and a plugin computing graph analysis should rely only on these properties. However, for rendering and visualization, this ``containment'' could be represented as a \cxt as well. If a specific Ondex graph is intended for visualization only, the properties may be omitted.

\item In general, \cxt should not be used to express \re, in particular ``containment'' (genes in chromosomes) or is-a (Terms in Gene Ontology associated to more generic ones.). We note that is-a should be expressed only by relations in between \term{ConceptClass}.

\item A \cxt should be a proper sub-graph, it should not contain itself, it should not contain \re without all \co s associated to that \re. It should also not be cyclical, i.e. a \cxt X should not contain \co Y whose associated \cxt s includes X.

\item When used to characterize a sub-graph in Ondex, a \cxt should be associated to a \co of type ``sub-graph'' (not to a generic \co).  A ``sub-graph'' cannot be used to characterize a \cxt containing itself. All the meta-elements of a ``sub-graph'' node are intended to reflect the set of nodes and edges that the related context include:
\begin{itemize}
\item \term{Name}: a label for the context
\item \term{ControlledVocabulary}: the most generic CV (provenance) for all elements this \cxt includes. Unless a \cxt is used to characterize the CV for a sub-graph, this would be a generic term.
\item \term{Evidence}: the most generic evidence for all elements this \cxt includes. Unless a \cxt is used to characterize evidence, this would be a generic term.
\item \term{ConceptClass}: ``sub-graph''
\item \term{ConceptAccession}: this can be used to characterize scope of information (e.g.: CV=temp, ID=1). More on this in the document on scope of information in Ondex.
\end{itemize}
\item The above metadata (\term{Name}, \term{ControlledVocabulary}, \term{Evidence}, \term{Accession}) for a  \cxt are implied by the same metadata for the \co s and \re s that this \cxt contains (and vice versa). It should not be assumed that is implication is computed by Ondex. Checking this constraints is possible, but not implemented.
\item a node of type ``sub-graph'' should not be related to another type of concept in Ondex (other than with the meaning of ``annotation'').


\end{itemize}



\subsection{Future development}
\begin{itemize}
\item a \cxt intended as a sub-graph should not be necessarily represented as a node or displayed.
\item relations between a ``sub-graph'' and another type of concept in Ondex could be possible, but a further analysis of their implication is required before they can be used in a sound way.
\end{itemize}


\end{document}

